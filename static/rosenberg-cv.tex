% Options for packages loaded elsewhere
\PassOptionsToPackage{unicode}{hyperref}
\PassOptionsToPackage{hyphens}{url}
%
\documentclass[
  14,
]{article}
\usepackage[includeheadfoot, margin=1in]{geometry}
\usepackage{fancyhdr}
\pagestyle{fancy}
\fancyhf{}
\fancyhead[R]{Joshua M. Rosenberg}
\lfoot{Updated \date{\today}}
\rfoot{\thepage}
\usepackage{lmodern}
\usepackage{amssymb,amsmath}
\usepackage{ifxetex,ifluatex}
\ifnum 0\ifxetex 1\fi\ifluatex 1\fi=0 % if pdftex
  \usepackage[T1]{fontenc}
  \usepackage[utf8]{inputenc}
  \usepackage{textcomp} % provide euro and other symbols
\else % if luatex or xetex
  \usepackage{unicode-math}
  \defaultfontfeatures{Scale=MatchLowercase}
  \defaultfontfeatures[\rmfamily]{Ligatures=TeX,Scale=1}
\fi
% Use upquote if available, for straight quotes in verbatim environments
\IfFileExists{upquote.sty}{\usepackage{upquote}}{}
\IfFileExists{microtype.sty}{% use microtype if available
  \usepackage[]{microtype}
  \UseMicrotypeSet[protrusion]{basicmath} % disable protrusion for tt fonts
}{}
\makeatletter
\@ifundefined{KOMAClassName}{% if non-KOMA class
  \IfFileExists{parskip.sty}{%
    \usepackage{parskip}
  }{% else
    \setlength{\parindent}{0pt}
    \setlength{\parskip}{6pt plus 2pt minus 1pt}}
}{% if KOMA class
  \KOMAoptions{parskip=half}}
\makeatother
\usepackage{xcolor}
\IfFileExists{xurl.sty}{\usepackage{xurl}}{} % add URL line breaks if available
\IfFileExists{bookmark.sty}{\usepackage{bookmark}}{\usepackage{hyperref}}
\hypersetup{
  pdftitle={Joshua M. Rosenberg},
  hidelinks,
  pdfcreator={LaTeX via pandoc}}
\urlstyle{same} % disable monospaced font for URLs
\usepackage{graphicx,grffile}
\makeatletter
\def\maxwidth{\ifdim\Gin@nat@width>\linewidth\linewidth\else\Gin@nat@width\fi}
\def\maxheight{\ifdim\Gin@nat@height>\textheight\textheight\else\Gin@nat@height\fi}
\makeatother
% Scale images if necessary, so that they will not overflow the page
% margins by default, and it is still possible to overwrite the defaults
% using explicit options in \includegraphics[width, height, ...]{}
\setkeys{Gin}{width=\maxwidth,height=\maxheight,keepaspectratio}
% Set default figure placement to htbp
\makeatletter
\def\fps@figure{htbp}
\makeatother
\setlength{\emergencystretch}{3em} % prevent overfull lines
\providecommand{\tightlist}{%
  \setlength{\itemsep}{0pt}\setlength{\parskip}{0pt}}
\setcounter{secnumdepth}{-\maxdimen} % remove section numbering

\title{Joshua M. Rosenberg\vspace{-3em}}

\author{}
\date{}

\begin{document}
\maketitle
\thispagestyle{empty}

\begin{center}

\section{Curriculum Vitae}\label{curriculum-vitae}

Assistant Professor, STEM Education\linebreak
Department of Theory and Practice in Teacher Education\linebreak
Faculty Fellow, Center for Enhancing Education in Mathematics and Sciences\linebreak
The University of Tennessee, Knoxville\linebreak
420 Claxton, 1122 Volunteer Blvd., Knoxville, TN 37996\linebreak
865-974-5973 | jmrosenberg@utk.edu\linebreak
Homepage: https://joshuamrosenberg.com\linebreak
Research Group: https://makingdatasciencecount.com\linebreak  
\end{center}

\hypertarget{research-interests}{%
\section{Research Interests}\label{research-interests}}

Science education, educational data science, science teacher education,
computer science education

\hypertarget{education}{%
\section{Education}\label{education}}

\hypertarget{degrees}{%
\subsubsection{Degrees}\label{degrees}}

2018, PhD, Educational Psychology \& Educational Technology\\
\emph{Committee}: Matthew J. Koehler (Co-chair), Jennifer A. Schmidt
(Co-chair), Lisa Linnenbrink-Garcia, and Christina V. Schwarz\\
Michigan State University

2012, MA, Education\\
Michigan State University

2010, BS, Biology\\
University of North Carolina, Asheville

\hypertarget{additional-qualifications}{%
\subsubsection{Additional
Qualifications}\label{additional-qualifications}}

2016, Graduate Certificate, Science Education\\
Michigan State University

2010, Teacher Licensure Program\\
University of North Carolina, Asheville

\hypertarget{professional-experience}{%
\section{Professional Experience}\label{professional-experience}}

2018-present, Assistant Professor, STEM Education\\
University of Tennessee, Knoxville

2012-2018, Graduate Research and Teaching Assistant\\
Michigan State University

2010-2012, Science Teacher (Biology and Earth Science)\\
Shelby High School, Shelby, NC

2009-2010, Science Teacher Intern (Biology and Chemistry)\\
C.D. Owen High School, Swannanoa, NC

\hypertarget{grants}{%
\section{Grants}\label{grants}}

2020-2025, Co-Investigator (Co-I), \emph{Imagining Possibilities in
Post-Secondary Education and STEMM in Rural Appalachia}. (\$1,208,563),
National Institutes of Health.

2020-2023, Senior Personnel, \emph{Learning analytics in STEM education
research institute} (\$993,150; \emph{PI}, Shaun Kellogg, North Carolina
State University; UTK subcontract: \$54,691). NSF. (NSF Grant
No.~\href{https://www.nsf.gov/awardsearch/showAward?AWD_ID=2025090\&HistoricalAwards=false}{2025090})

2020-2021, Co-Principal Investigator (Co-PI), \emph{Propelling teacher
professional development through FAAST feedback on student epistemic
views} (\$15,000; \emph{PI}: Christina Krist, University of Illinois
Urbana-Champaign). Technology Innovations in Educational Research and
Design Pilot Projects Program.

2019-2021, PI, \emph{Understanding the development of interest in
computer science: An experience sampling approach} (\$346,688). National
Science Foundation {[}NSF{]}. \url{http://picsul.utk.edu/} (NSF Grant
No.~\href{https://www.nsf.gov/awardsearch/showAward?AWD_ID=1937700\&HistoricalAwards=false}{1937700})

2019-2021, Co-PI, \emph{CS for Appalachia: A research-practice
partnership for integrating computer science into East Tennessee
schools} (\$252,453; \emph{PI}: Lynn Hodge, University of Tennessee,
Knoxville). NSF. (NSF Grant
No.~\href{https://www.nsf.gov/awardsearch/showAward?AWD_ID=1923509\&HistoricalAwards=false}{1923509})

2019-2022, Co-PI, \emph{Advancing computational grounded theory for
audiovisual data from STEM classrooms} (\$1,313,855; \emph{PI}:
Christina Krist, University of Illinois Urbana-Champaign; University of
Tennessee, Knoxville {[}UTK{]} subcontract: \$101,469). NSF.
\url{https://tca2.education.illinois.edu/} (NSF Grant
No.~\href{https://www.nsf.gov/awardsearch/showAward?AWD_ID=1920796\&HistoricalAwards=false}{1920796})

2019-2022, Senior Personnel, \emph{Medical entomology and geospatial
analyses: Bringing innovation to teacher education and surveillance
studies} (\$149,611; \emph{PI}: Rebecca Trout Fryxell). United States
Department of Agriculture - Agriculture and Food Research Initiative.
(USDA Grant No.~2019-68010-29119) \url{https://www.megabitess.org/}

2019-2020, PI, \emph{Planting the seeds for computer science education
in East Tennessee through a research-practice partnership} (\$13,200).
Community Engaged Research Seed Program, UTK.

2018-2020, Co-PI, \emph{Exploring how beginning elementary mathematics
teachers seek out resources through social media} (\$8,820; \emph{PI}:
Stephen Aguilar). Herman \& Rasiej K-5 Mathematics Initiative,
University of Southern California.

2013, PI, \emph{Basic biology for everyone} (\$2,000), Versal Foundation
Grant

\hypertarget{fellowships-and-awards}{%
\section{Fellowships and Awards}\label{fellowships-and-awards}}

2021, Louie M. \& Betty M. Phillips Faculty Support in Education Award

2021, Mentor, Summer Undergraduate Research Internship Program, Office
of Undergraduate Research, University of Tennessee, Knoxville (UTK)

2020, Research Assistant Award, Office of Undergraduate Research,UTK

2020, Southeastern Conference (SEC) Visiting Faculty Travel Grant
Program (Host: Annelise Russell, Martin School of Public Policy,
University of Kentucky)

2019-2020, Initiative for the Future Faculty Development Program, UTK

2019, Finalist, Association for Science Teacher Education John C. Park
National Technology Leadership Institute Fellowship

2017, Delia Koo Global Travel Fellowship, Michigan State University
(MSU)

2017, Michigan Virtual Learning Research Institute Dissertation
Fellowship

2017, Concord Consortium Data Science Educational Technology Fellowship

2017, Council of Graduate Students Disciplinary Leadership Award, MSU

2016, College of Education Alumni Fellowship, MSU

2016, Best Paper Award, Technological Pedagogical Content Knowledge
Special Interest Group (SIG), Society for Information Technology and
Teacher Education International Conference

2015, Cotterman Family Endowment for Education Summer Research
Fellowship, MSU

2014, Outstanding Paper Award, Society for Information Technology and
Teacher Education International Conference

2013, Massive Open Online Course Research and Development Fellowship,
MSU

2009-2012, Burroughs Wellcome Fund Scholar, University of North
Carolina, Asheville

\hypertarget{publications}{%
\section{Publications}\label{publications}}

\hypertarget{articles-published-in-refereed-journals}{%
\subsection{Articles Published in Refereed
Journals}\label{articles-published-in-refereed-journals}}

\hangindent=2em Rosenberg, J. M., Borchers, C., Dyer, E., Anderson, D.
J., \& Fischer, C. (in press). Advancing new methods for understanding
public sentiment about educational reforms: The case of Twitter and the
Next Generation Science Standards. \emph{AERA Open}.
\url{https://osf.io/xymsd/}

\hangindent=2em Rutherford, T., Duck, K., Rosenberg, J. M., \& Patt, R.
(in press). Leveraging mathematics software data to understand student
learning and motivation during the COVID-19 pandemic. \emph{Journal of
Research on Technology in Education}.

\hangindent=2em Greenhalgh, S. P., Rosenberg, J. M., \& Russell, A. (in
press). The influence of policy and context on teachers' social media
use. \emph{The British Journal of Educational Technology}.

\hangindent=2em Schweinsburg, M., \ldots{} Luis, S. (in press). Same
data, different conclusions: Radical dispersion in empirical results.
\emph{Organizational Behavior and Human Decision Processes.}

\hangindent=2em Frank, K. A., Lin, Q., Maroulis, S., Strassman, A., Xu,
R., Rosenberg, J. M., Hayter, C., Mahmoud, R., Kolak, M., Dietz, T., \&
Zhang, L. (advance online publication). Hypothetical case replacement
can be used to quantify the robustness of trial results. \emph{Journal
of Clinical Epidemiology}.
\url{https://www.sciencedirect.com/science/article/pii/S0895435621000366}
(\emph{N.B.:} I was a scientific programmRer for this project).

\hangindent=2em Ranellucci, J., Robinson, K., Rosenberg, J. M., Lee,
Y.-K., Roseth, C., \& Linnenbrink-Garcia. (advance online publication).
Comparing the roles and correlates of emotions in class and during
online video lectures in a flipped anatomy classroom.
\url{https://doi.org/10.1016/j.cedpsych.2021.101966}

\hangindent=2em Akcaoglu, M., Rosenberg, J. M., Hodges, C. B., Hilpert,
J. (2021). An exploration of factors impacting middle school students'
attitudes toward computer programming. \emph{Computers in the Schools.
38}(1), 19-35. \url{https://doi.org/10.1080/07380569.2021.1882209}

\hangindent=2em Rosenberg, J. M., \& Krist, C. (2021). Combining machine
learning and qualitative methods to elaborate students' ideas about the
generality of their model-based explanations. Journal of Science
Education and Technology, 30(2), 255-267.
\url{https://link.springer.com/article/10.1007\%2Fs10956-020-09862-4}.
\emph{Nb.} Both authors contributed equally.

\hangindent=2em Rosenberg, J. M., \& Staudt Willet, K. B. (2021).
Balancing participants' privacy and open science in the context of
COVID-19: A response to Ifenthaler \& Schumacher (2016).
\emph{Educational Technology Research \& Development, 69}(1), 347-351.

\hangindent=2em Harper, F. K., Rosenberg, J. M., Comperry, S., Howell,
K., \& Womble, S. (2021). \#Mathathome during the COVID-19 Pandemic:
Exploring and reimagining resources and social supports for parents.
\emph{Education Sciences, 11}(2), 60, 1-24.
\url{https://www.mdpi.com/2227-7102/11/2/60}

\hangindent=2em Anderson, D. J., Rowley, B., Stegenga, S., Irvin, P. S.,
\& Rosenberg, J. M. (2020). Evaluating content-related validity evidence
using a text-based, machine learning procedure. \emph{Educational
Measurement: Issues and Practice, 39}(4), 53-64.
\url{https://onlinelibrary.wiley.com/doi/abs/10.1111/emip.12314}

\hangindent=2em Rosenberg, J. M., Reid, J., Dyer, E., Koehler, M. J.,
Fischer, C., \& McKenna, T. J. (2020). Idle chatter or compelling
conversation? The potential of the social media-based \#NGSSchat network
as a support for science education reform efforts. \emph{Journal of
Research in Science Teaching, 57}(9), 1322-1355.
\url{https://onlinelibrary.wiley.com/doi/10.1002/tea.21660}

\hangindent=2em Carpenter, J., Rosenberg, J. M., Dousay, T.,
Romero-Hall, E., Trust, T., Kessler, A., Phillips, M., Morrison, S.,
Fischer, C. \& Krutka, D. (2020). What should teacher educators know
about technology? Perspectives and self-assessments. \emph{Teaching and
Teacher Education, 95}(10), 103-124.
\url{https://doi.org/10.1016/j.tate.2020.103124}

\hangindent=2em Ranellucci, J., Rosenberg, J. M., \& Poitras, E. (2020).
Exploring pre-service teachers' use of technology: The technology
acceptance model and expectancy-value theory. \emph{Journal of Computer
Assisted Learning, 36}(6), 810-824.
\url{http://dx.doi.org/10.1111/jcal.12459}

\hangindent=2em Schmidt, J. A., Beymer, P. N., Rosenberg, J. M.,
Naftzger, N. J., \& Shumow, L. (2020). Experiences, Activities, and
personal characteristics as predictors of engagement in STEM-focused
summer programs. \emph{Journal of Research in Science Teaching, 57}(8),
1281-1309.
\url{https://onlinelibrary.wiley.com/doi/full/10.1002/tea.21630}

\hangindent=2em Greenhalgh, S. P., Rosenberg, J. M., Koehler, M. J.,
Akcaoglu, M., \& Staudt Willet, K. B. (2020). Identifying multiple
learning spaces within a single teacher-focused Twitter hashtag.
\emph{Computers \& Education, 148}(4).
\url{https://doi.org/10.1016/j.compedu.2020.103809}

\hangindent=2em Beymer, P. N., Rosenberg, J. M., \& Schmidt, J. A.
(2020). Does choice matter or is it all about interest? An investigation
using an experience sampling approach in high school science classrooms.
\emph{Learning and Individual Differences, 78}(2), 1-15.
\url{https://doi.org/10.1016/j.lindif.2019.101812}

\hangindent=2em Rosenberg, J. M., Edwards, A., \& Chen, B. (2020).
Getting messy with data: Tools and strategies to help students analyze
and interpret complex data sources. \emph{The Science Teacher, 87}(5).
\url{https://learningcenter.nsta.org/resource/?id=10.2505/4/tst20_087_05_30}

\hangindent=2em Xu, R., Frank, K. A., Maroulis, S., \& Rosenberg, J. M.
(2019). konfound: Command to quantify robustness of causal inferences.
\emph{The Stata Journal, 19}(3), 523-550.
\url{https://journals.sagepub.com/doi/full/10.1177/1536867X19874223}

\hangindent=2em Blondel, D. V., Sansone, A., Rosenberg, J. M., Yang, B.
W., Linennbrink-Garcia, L., \& Schwarz-Bloom, R. D. (2019). Development
of an online experiment platform (Rex) for high school biology.
\emph{Journal of Formative Design for Learning, 3}(1) 62-81.
\url{https://www.ncbi.nlm.nih.gov/pmc/articles/PMC6716597/}

\hangindent=2em Henriksen, D., Mehta, R. \& Rosenberg, J. (2019).
Supporting a creatively focused technology fluent mindset among
educators: survey results from a five-year inquiry into teachers'
confidence in using technology. \emph{Journal of Technology and Teacher
Education, 27}(1), 63-95.
\url{https://www.learntechlib.org/primary/p/184724/}

\hangindent=2em Rosenberg, J. M., \& Lawson, M. J. (2019). An
investigation of students' use of a computational science simulation in
an online high school physics class. \emph{Education Sciences, 9}(49),
1-19. \url{https://www.mdpi.com/2227-7102/9/1/49}

\hangindent=2em Rosenberg, J. M., Beymer, P. N., Anderson, D. J., \&
Schmidt, J. A. (2018). tidyLPA: An R package to easily carry out Latent
Profile Analysis (LPA) using open-source or commercial software.
\emph{Journal of Open Source Software, 3}(30), 978.
\url{https://doi.org/10.21105/joss.00978}

\hangindent=2em Greenhalgh, S. P., Staudt Willet, K. B., Rosenberg, J.
M., \& Koehler, M. J. (2018). Tweet, and we shall find: Using digital
methods to locate participants in educational hashtags.
\emph{TechTrends, 62}(5), 501-508.
\url{https://doi.org/10.1007/s11528-018-0313-6}

\hangindent=2em Beymer, P. N., Rosenberg, J. M., Schmidt, J. A., \&
Naftzger, N. (2018). Examining relationships between choice, affect, and
engagement in out-of-school time STEM programs. \emph{Journal of Youth
and Adolescence, 47}(6), 1178-1191.
\url{https://doi.org/10.1007/s10964-018-0814-9}

\hangindent=2em Akcaoglu, M., Rosenberg, J. M., Ranellucci, J., \&
Schwarz, C. V. (2018). Outcomes from a self-generated utility value
intervention on fifth and sixth-grade students' value and interest in
science. \emph{International Journal of Educational Research, 87},
67-77.
\url{https://www.sciencedirect.com/science/article/pii/S0883035517308492}

\hangindent=2em Schmidt, J. A., Rosenberg, J. M., \& Beymer, P. (2018).
A person-in-context approach to student engagement in science: Examining
learning activities and choice. \emph{Journal of Research in Science
Teaching, 55}(1), 19-43. \url{https://dx.doi.org/10.1002/tea.21409}
(\emph{Nb. This article was recognized as one of the 20 most-downloaded
articles in JRST between June, 2016 and June, 2018})

\hangindent=2em Rosenberg, J.M., Greenhalgh, S.P., Graves Wolf, L. \&
Koehler, M.J. (2017). Strategies, use, and impact of social media for
supporting teacher community within professional development: The case
of one urban STEM program. \emph{Journal of Computers in Mathematics and
Science Teaching, 36}(3), 255-267.
\url{https://www.learntechlib.org/primary/p/180387/}

\hangindent=2em Koehler, M. J., Greenhalgh, S. P., Rosenberg, M. J., \&
Keenan, S. (2017). What the tech is going on with digital teaching
portfolios? Using the TPACK framework to analyze teachers' technological
understanding. \emph{Journal of Technology and Teacher Education, 25},
31-59. \url{http://www.learntechlib.org/p/173346/} "\textgreater{}

\hangindent=2em Rosenberg, J. M., Greenhalgh, S. P., Koehler, M. J.,
Hamilton, E., \& Akcaoglu, M. (2016). An investigation of State
Educational Twitter Hashtags (SETHs) as affinity spaces.
\emph{E-Learning and Digital Media, 13}(1-2), 24-44.
\url{http://dx.doi.org/10.1177/2042753016672351}

\hangindent=2em Greenhalgh, S. P., Rosenberg, J. M., \& Wolf, L. G.
(2016). For all intents and purposes: Twitter as a foundational
technology for teachers. \emph{E-Learning and Digital Media, 13}(1-2),
81-98. \url{http://dx.doi.org/10.1177/2042753016672131}

\hangindent=2em Hamilton, E., Rosenberg, J. M., \& Akcaoglu, M. (2016).
Examining the Substitution Augmentation Modification Redefinition (SAMR)
model for technology integration. \emph{Tech Trends, 60}, 433-441.
\url{http://dx.doi.org/10.1007/s11528-016-0091-y}

\hangindent=2em Rosenberg, J. M., Terry, C. A., Bell, J., Hiltz, V., \&
Russo, T. (2016). Design guidelines for graduate program social media
use. \emph{Tech Trends, 2}, 167-175.
\url{http://dx.doi.org/10.1007/s11528-016-0023-x}

\hangindent=2em Rosenberg, J. M., \& Koehler, M. J. (2015). Context and
Technological Pedagogical Content Knowledge (TPACK): A systematic
review. \emph{Journal of Research on Technology in Education, 47},
186-210. \url{http://dx.doi.org/10.1080/15391523.2015.1052663}

\hypertarget{working-papers-and-papers-under-review}{%
\subsection{Working Papers and Papers Under
Review}\label{working-papers-and-papers-under-review}}

\emph{If a link to a pre-print is not available, then please contact me
to request a copy of any of these papers.}

Building Trust in Science by Advancing a Bayesian Perspective on
Probability and Uncertainty in Science Education.
\url{https://osf.io/aznyq/}

Me, the moment, or the medium? Understanding the sources of variation
for students' situational engagement in science.
\url{https://osf.io/pj2v8/}

Open for whom? A Call for a science education-specific open science.
\url{https://osf.io/sqcn7/}

Characterizing whole class discussions about data and statistics with
conversation profile analysis.

Better together: Understanding and supporting math and science
pre-service teachers' sensemaking about STEM.

Posts on Facebook by Schools and Districts and the Potential Risks to
Students' Privacy. \url{https://osf.io/ujyg2/}

\hypertarget{book}{%
\subsection{Book}\label{book}}

\hangindent=2em Estrellado, R. A., Freer, E. A., Mostipak, J.,
Rosenberg, J. M., \& Velásquez, I. C. (2020). \emph{Data science in
education using R}. London, England: Routledge. \emph{Nb.} All authors
contributed equally. \url{http://www.datascienceineducation.com/}

\hypertarget{contributions-to-edited-volumes}{%
\subsection{Contributions to Edited
Volumes}\label{contributions-to-edited-volumes}}

\hangindent=2em Dai, T., Rosenberg, J. M., \& Lawson, M. A. (in press).
Data representation. In T. L. Good \& M. McCaslin (\emph{Eds.}),
Educational Psychology Section; D. Fisher (\emph{Ed.}), \emph{Routledge
encyclopedia of education} (Online). Taylor \& Francis: New York, NY.

\hangindent=2em Rosenberg, J. M., Lawson, M. A., Anderson, D. J., \&
Rutherford, T. (2020). Making data science count in and for education.
In E. Romero-Hall (Ed.), \emph{Research methods in learning design \&
technology} (pp.~94-110). Routledge: New York, NY.

\hangindent=2em Greenhalgh, S. P., Staudt Willet, B., Rosenberg, J. M.,
\& Koehler, M. J. (2020). Lessons learned from applying Twitter research
methods to educational technology phenomena. In E. Romero-Hall (Ed.),
\emph{Research methods in learning design \& technology} (pp.~64-77).
Routledge: New York, NY.

\hangindent=2em Eidelman, R., Rosenberg, J. M. ,\& Shwartz, Y. (2019).
Assessing the interaction between self-regulated learning (SRL) profiles
and actual learning in the chemistry online blended learning environment
(COBLE). In Sampson, D., D. Ifenthaler, M. Spector, P. Isafas, \& S.
Sergis (Eds), \emph{Learning technologies for transforming teaching,
learning and assessment at large scale} (pp.~231-255). Berlin, Germany:
Springer.

\hangindent=2em Herring, M., Koehler, M. J., Mishra, P., Rosenberg, J.
M., \& Teske, J. (2016). Introduction to the 2nd edition of the TPACK
handbook. In M. Herring, M. J. Koehler, \& P. Mishra (Eds.),
\emph{Handbook of Technological Pedagogical Content Knowledge (TPACK)
for educators} (2nd ed., pp.~1-8). New York, NY: Routledge.

\hangindent=2em Keenan, S., Rosenberg, J. M., Greenhalgh, S. P. \&
Koehler, M. J. (2016). Examining teachers' technology use through
digital portfolios. In L. Liu \& D. C. Gibson (Eds.), \emph{Research
highlights in technology and teacher education 2016} (pp.~53-60).
Chesapeake, VA: Association for the Advancement of Computing in
Education.

\hangindent=2em Phillips, M., Koehler, M. J. \& Rosenberg, J. M. (2016).
Considering context: Teachers' TPACK development and enactment. In L.
Liu \& D. C. Gibson (Eds.), \emph{Research highlights in technology and
teacher education} (pp.~197-204). Chesapeake, VA: Association for the
Advancement of Computing in Education. "\textgreater{}

\hangindent=2em Rosenberg, J. M., \& Koehler, M. J. (2015).
\emph{Context and teaching with technology in the digital age}. In M.L.
Niess \& H. Gillow-Wiles (Eds.), Handbook of research on teacher
education in the digital age (pp.~440-465). Hershey, PA: IGI Global.

\hangindent=2em Rosenberg, J. M., Greenhalgh, S. P., \& Koehler, M. J.
(2015). A performance assessment of teachers' TPACK using artifacts from
digital portfolios. In L. Liu \& D. C. Gibson, \emph{Research highlights
in technology and teacher education 2015}. Waynesville, NC: Association
for the Advancement of Computing in Education (AACE).

\hangindent=2em Koehler, M. J., Mishra, P., Akcaoglu, M., \& Rosenberg,
J. M. (2013). Technological pedagogical content knowledge for teachers
and teacher educators. In N. Bharati and S. Mishra (Eds.), \emph{ICT
integrated teacher education: A resource book} (pp.~1-8). Commonwealth
Educational Media Center for Asia, New Delhi, India.

\hypertarget{papers-published-in-refereed-conference-proceedings}{%
\subsection{Papers Published in Refereed Conference
Proceedings}\label{papers-published-in-refereed-conference-proceedings}}

\hangindent=2em Rosenberg, J. M., \& Kubsch, M. (2021). Advancing K-12
learners' use of Bayesian methods. In XXX (Eds.), \emph{The
International Society of the Learning Sciences 2020 Conference
Proceedings} (pp.~xx-xx). International Society of the Learning
Sciences.

\hangindent=2em Rosenberg, J. M., Galas, E., \& Staudt Willet, K.B.
(2021). Who are the data scientists in education? An investigation of
the identities and work of individuals in diverse roles.In XXX (Eds.),
\emph{The International Society of the Learning Sciences 2020 Conference
Proceedings} (pp.~xx-xx). International Society of the Learning
Sciences.

\hangindent=2em Kubsch, M., Rosenberg. J. M., \& Krist, C. (2021).
Beyond supervision: Human / machine distributed learning in learning
sciences research. In XXX (Eds.), \emph{The International Society of the
Learning Sciences 2020 Conference Proceedings} (pp.~xx-xx).
International Society of the Learning Sciences.

\hangindent=2em Rosenberg, J. M., \& Nguyen, H. (2021). How K-12 school
districts communicated during the COVID-19 pandemic: A study using
Facebook data. \emph{Proceedings of the 11th International Conference on
Learning Analytics \& Knowledge (LAK21)}.

\hangindent=2em Rosenberg, J. M., \& Staudt Willet, K. B. (2021).
Advancing social influence models in learning analytics. \emph{Companion
proceedings of the 11th International Conference on Learning Analytics
\& Knowledge (LAK21)}

\hangindent=2em Lishinski, A., \& Rosenberg, J. M. (2021). How CS1
students experienced COVID-19 in the moment: using an experience
sampling approach to understand the transition to emergency remote
instruction. In \emph{Proceedings of the 51st ACM Technical Symposium on
Computer Science Education} (pp.~1414-1414).

\hangindent=2em Mann, M., Bui, H., Gibbons, B., Lishinski, A., Dyer, E.,
\& Rosenberg, J. M. (2021). ``Not my subject''?: A survey of teachers
regarding the implementation of new K-8 computing education standards.
In \emph{Proceedings of the 51st ACM Technical Symposium on Computer
Science Education} (pp.~1414-1414).

\hangindent=2em Rosenberg, J. M., Schmidt, A., Rosenberg, A. M.,
Longnecker, J., \& Mann M. (2020). Becoming `tidier' over time. Studying
\#tidytuesday as a social media-based context for learning to visualize
data. In M. Gresalfi and I. Horn (Eds.), \emph{The Interdisciplinarity
of the Learning Sciences: The International Conference of the Learning
Sciences 2020 Conference Proceedings} (Vol 3., pp.~1811-1812). ISLS.

\hangindent=2em Jones R. S., \& Rosenberg, J. M. (2020). Studying whole
class discussions at scale. In M. Gresalfi and I. Horn (Eds.), \emph{The
Interdisciplinarity of the Learning Sciences: The International
Conference of the Learning Sciences 2020 Conference Proceedings} (Vol
5., pp.~2499-2506). ISLS.

\hangindent=2em D'Angelo, C., Dyer, E. B., Krist, C., Rosenberg, J. M.,
\& Bosch, N. (2020). Advancing computational grounded theory for
audiovisual data from mathematics classrooms. In M. Gresalfi and I. Horn
(Eds.), \emph{The Interdisciplinarity of the Learning Sciences: The
International Conference of the Learning Sciences 2020 Conference
Proceedings} (Vol 4., pp.~2393-2394). ISLS.

\hangindent=2em Dyer, E. B., D'Angelo, D., Bosch, N., Krist, C., \&
Rosenberg, J. M. (2020). Analyzing learning with speech analytics and
computer vision methods: Technologies, principles, and ethics. In M.
Gresalfi and I. Horn (Eds.), \emph{The Interdisciplinarity of the
Learning Sciences: The International Conference of the Learning Sciences
2020 Conference Proceedings} (Vol 5., pp.~2651-2652). ISLS.

\hangindent=2em Rosenberg, J. M. (2020). More confidently uncertain?
Teaching learners to apply Bayesian methods to make sense of scientific
phenomena. In M. Gresalfi and I. Horn (Eds.), \emph{The
Interdisciplinarity of the Learning Sciences: The International
Conference of the Learning Sciences 2020 Conference Proceedings} (Vol
5., pp.~2681-2682). ISLS.

\hangindent=2em Lishinski, A., \& Rosenberg, J. (2020, February).
Accruing interest: What experiences contribute to students developing a
sustained interest in computer science over time? In \emph{Proceedings
of the 51st ACM Technical Symposium on Computer Science Education}
(pp.~1414-1414).

\hangindent=2em Rosenberg, J., \& Lishinski, A. (2020, February).
Variable interest rate: What experiences explain differences in interest
in computer science among students? In \emph{Proceedings of the 51st ACM
Technical Symposium on Computer Science Education} (pp.~1404-1404).

\hangindent=2em Carpenter, J., Rosenberg, J. M., Dousay, T.,
Romero-Hall, E., Trust, T., Kessler, A., Phillips, M., Morrison, S.,
Fischer, C. \& Krutka, D. (2019). What do teacher educators think of
teacher education technology competencies?. In K. Graziano (Ed.),
\emph{Proceedings of Society for Information Technology \& Teacher
Education International Conference} (pp.~796-801). Las Vegas, NV, United
States: Association for the Advancement of Computing in Education
(AACE). Retrieved April 18, 2019 from
\url{https://www.learntechlib.org/primary/p/207735/}.

\hangindent=2em Peterson, A., Freer, D., \& Rosenberg, J. M. (2017).
Interacting with purpose: What is the difference between face-to-face
and online student relationships in a combined program? In
\emph{Proceedings of Society for Information Technology \& Teacher
Education International Conference} (pp.~3411-3414). Austin, TX:
Association for the Advancement of Computing in Education. Retrieved
from \url{https://www.learntechlib.org/p/177955/}

\hangindent=2em Krist, C., \& Rosenberg, J. M. (2016). Finding patterns
in and refining characterizations of students' epistemic cognition: A
computational approach. In Looi, C.-K., Polman, J., Cress, U., \&
Reimann, P. (Eds.), \emph{Transforming Learning, Empowering Learners:
The International Conference of the Learning Sciences Proceedings} 2016
(Vol. 2, pp.~1223-1224). Singapore, Singapore: ICLS.

\hangindent=2em Rosenberg, J. M., Koehler, M. J., Akcaoglu, M.,
Greenhalgh, S. P. \& Hamilton, E. (2016). State Educational Twitter
Hashtags: An introduction and research agenda. In \emph{Proceedings of
Society for Information Technology \& Teacher Education International
Conference 2016} (pp.~355-360). Chesapeake, VA: Association for the
Advancement of Computing in Education. Retrieved from
\url{http://www.editlib.org/p/171698}

\hangindent=2em Greenhalgh, S. P., Rosenberg, J. M. \& Wolf, L. G.
(2016). For every tweet there is a purpose: Twitter within (and beyond)
an online graduate program. In \emph{Proceedings of Society for
Information Technology \& Teacher Education International Conference
2016} (pp.~2044-2049). Chesapeake, VA: Association for the Advancement
of Computing in Education (AACE). Retrieved from
\url{http://www.editlib.org/p/171972}

\hangindent=2em Schwarz, C. V., Ke, L., Lee, M, \& Rosenberg, J. M.
(2014). Developing mechanistic explanations of phenomena: Case studies
of two fifth grade students' epistemologies in practice over time. In J.
L. Polman, E. A. Kyza, K. O'Neill, I. Tabak, W. R. Penuel, A. S. Jurow,
. . . L. D'Amico (Eds.), \emph{Learning and becoming in practice: The
international conference of the learning sciences (ICLS) 2014} (Vol. 1,
pp.~182-189). Boulder, CO: ISLS.
\url{http://www.isls.org/icls2014/Proceedings.html} "\textgreater{}

\hangindent=2em Rosenberg, J. M., \& Koehler, M. (2014). Context and
Technological Pedagogical Content Knowledge: A content analysis. In M.
Searson \& M. Ochoa (Eds.), \emph{Proceedings of Society for Information
Technology \& Teacher Education International Conference 2014}
(pp.~2412-2417). Chesapeake, VA: AACE. Retrieved from
\url{http://www.editlib.org/p/131183}

\hangindent=2em Greenhalgh, S. P., Rosenberg, J. M., Zellner, A. \&
Koehler, M. J. (2014). Zen and the art of portfolio maintenance: Best
practices in course design for supporting long-lasting portfolios. In M.
Searson \& M. Ochoa (Eds.), \emph{Proceedings of Society for Information
Technology \& Teacher Education International Conference 2014}
(pp.~1604-1610). Chesapeake, VA: AACE. Retrieved from
\url{http://www.editlib.org/p/131027}

\hangindent=2em Rosenberg, J., Terry, C., Bell, J., Hiltz, V., Russo, T.
\& The EPET Social Media Council (2014). What we've got here is failure
to communicate: Social media best practices for graduate school
programs. In M. Searson \& M. Ochoa (Eds.), \emph{Proceedings of Society
for Information Technology \& Teacher Education International Conference
2014} (pp.~1210-1215). Chesapeake, VA: AACE. Retrieved from
\url{http://www.editlib.org/p/130949}

\hangindent=2em Rosenberg, J. (2013). Review of mobile device use
policies in public high schools. In R. McBride \& M. Searson (Eds.),
\emph{Proceedings of Society for Information Technology \& Teacher
Education International Conference 2013} (pp.~3774-3779). Chesapeake,
VA: AACE. Retrieved from \url{http://www.editlib.org/p/48698git}

\hypertarget{articles-published-in-non-refereed-journals}{%
\subsection{Articles Published in Non-Refereed
Journals}\label{articles-published-in-non-refereed-journals}}

\hangindent=2em Kimmons, R., Rosenberg, J., \& Allman, B. (2021). Trends
in Educational Technology: What Facebook, Twitter, and Scopus Can Tell
us about Current Research and Practice. \emph{TechTrends}, 1-12.
\url{https://link.springer.com/article/10.1007/s11528-021-00589-6}

\hangindent=2em Naftzger, N., Schmidt, J. A., Shumow, L., Beymer, P. N.,
\& Rosenberg, J. M. (2019). \emph{Exploring the link between STEM
activity leader practice and youth engagement: Findings from the STEM IE
study}. Washington, DC: American Institutes for Research.
\url{https://www.informalscience.org/exploring-link-between-stem-activity-leader-practice-and-youth-engagement-findings-stem-ie-study}

\hangindent=2em Mehta, R., Henriksen, D., \& Rosenberg, J. M. (2019).
It's not about the tools. \emph{Educational Leadership, 76}(5), 64-69.
Retrieved from
\url{http://www.ascd.org/publications/educational-leadership/feb19/vol76/num05/It's-Not-About-the-Tools.aspx}
\textless a href=``''\textgreater{}

\hangindent=2em Vo, T., \& Rosenberg, J. M. (2018). Posts for the NARST
Graduate Student Resources blog {[}four blog posts{]}. \emph{NARST
Graduate Student Resources Blog}. Linked through this page:
\url{https://joshuamrosenberg.com/job-market-resources.html}

\hangindent=2em Rosenberg, J. M. (2018). Opportunities for engaging
students in ``data practices'' in online science classes. \emph{Michigan
Virtual Learning Research Institute Blog: Research, Policy, Innovation
\& Networks}.
\url{https://mvlri.org/blog/opportunities-engaging-students-data-practices-online-science-classes/}

\hangindent=2em Rosenberg, J. M., \& Logan, C. W. (2017). Review of the
book What's Worth Teaching: Rethinking Curriculum in the Age of
Technology, by A. Collins. \emph{Teachers College Record}.
\url{http://www.tcrecord.org/Content.asp?ContentID=22173}

\hangindent=2em Phillips, M., Harris, J., Rosenberg, J. M., \& Koehler,
M. J. (2017). TPCK/TPACK research and development: Past, present, and
future directions. \emph{Australasian Journal of Educational
Technology}. \url{https://doi.org/10.14742/ajet.3907}

\hangindent=2em Rosenberg, J. M., \& Ranellucci, J. (2017). Student
motivation in online science courses: A path to spending more time on
course and higher achievement. \emph{Michigan Virtual Learning Research
Institute Blog: Research, Policy, Innovation \& Networks}.
\url{https://mvlri.org/blog/student-motivation-in-online-science-courses-a-path-to-spending-more-time-on-course-and-higher-achievement/}

\hypertarget{unpublished-manuscripts}{%
\subsection{Unpublished Manuscripts}\label{unpublished-manuscripts}}

\hangindent=2em Rosenberg, J. (2019). \emph{Understanding the
development of interest in CS: An experience sampling approach (Proposal
to NSF 19-4327)}. \url{https://doi.org/10.31219/osf.io/9mg5y} \emph{Nb.}
This proposal is associated with NSF Grant no.
\href{https://www.nsf.gov/awardsearch/showAward?AWD_ID=1937700\&HistoricalAwards=false}{1937700}.

\hangindent=2em Nosek, B. A., Ofiesh, L., Grasty, F. L., Pfeiffer, N.,
Mellor, D. T., Brooks, R. E., III, . . . Baraniuk, R. (2019).
\emph{Proposal to NSF 19-565 to create a STEM education research hub}.
\url{https://doi.org/10.31222/osf.io/4mpuc} \emph{Nb}. Contributors are
listed in alphabetical order following the Principal Investigators. I
contributed to this proposal.

\hangindent=2em Rosenberg, J. M. (2018). \emph{Context and Technological
Pedagogical Content Knowledge: An initial survey of teachers and
validation data}.

\hangindent=2em Rosenberg, J. M. (2018). \emph{Understanding work with
data in summer STEM programs: An experience sampling method approach}
(Doctoral dissertation). Retrieved from Proquest Dissertations and
Theses. (Proquest No.~10747232)

\hypertarget{presentations}{%
\section{Presentations}\label{presentations}}

\hypertarget{peer-reviewed-conference-presentations}{%
\subsection{Peer-Reviewed Conference
Presentations}\label{peer-reviewed-conference-presentations}}

\hangindent=2em Rosenberg, J. M., Carpenter, J., Michela, E., Sultana,
O., McKenna, T. J., Dyer, E. D., \& Reid, J. (2021, April). \emph{``Best
P.D. out there''? An exploration of the \#NGSSchat network on Twitter}.
Presentation at the American Educational Research Association Annual
Meeting.

\hangindent=2em Lishinski, A., Rosenberg, J. M., Sultana, O., Mann, M.,
\& Dunn, J. (2021, April). \emph{A text messaging--based experience
sampling method study of students' interest in introductory computer
science}. Presentation at the American Educational Research Association
Annual Meeting.

\hangindent=2em Michela, E., Rosenberg, J. M., Sultana, O., Burchfield,
M., Thomas, T., \& Kimmons, R. (2021, April). \emph{``Life will
eventually get back to normal'': School districts' Twitter use in
response to COVID-19}. Poster presentation at the American Educational
Research Association Annual Meeting.

\hangindent=2em Rosenberg, J. M., Borchers, C., Gibbons, B., Dyer, E.
D., Anderson, D. A., \& Fischer, C. (2020, April). Don't worry, be
happy: A sentment analysis of the \#NGSSchat Twitter Symposium
community. In M. Rehm. (Chair), \emph{Social opportunity spaces: How
social media can inform/shape educational policy processes}. Symposium
conducted at the American Educational Research Association Annual
Meeting.

\hangindent=2em Lawson, M. A., Rosenberg, J. M., \& Herrick, I. (2021,
April). \emph{Better together: Supporting and understanding preservice
teacher (PST) sense-making about STEM}. Presentation at the American
Educational Research Association Annual Meeting.

\hangindent=2em Kellogg, S., Jiang, S., Rosenberg, J. M., \& Moore, R.
(2021, April). Learning Analytics in STEM Education Research (LASER)
Institute. In F. J. Levine \& G. L. Wimberly (Chairs), \emph{Building
capacity in STEM education research: A discussion with directors of the
NSF institutes in research methods}. Symposium at the American
Educational Research Association Annual Meeting.

\hangindent=2em Schmidt, J.A., Schell, M.J., Beymer, P.N., Alberts,
K.M., Phun, V., Lee, M. \& Rosenberg, J.M. (2020, August).
\emph{Students' momentary science engagement predicts end-of-course
achievement}. Poster presented at the annual meetings of the American
Psychological Association. Washington, DC. (Conference canceled)

\hangindent=2em Rosenberg, J. M., Reid, J., Dyer, E., Koehler, M. J.,
Fischer, C., \& McKenna, T. J. (2020, April). A new context for
professional networks: Understanding the social structure of \#NGSSChat
through social network analysis. In K. A. Frank, K., Torphy, K., Daly,
A., \& Greenhow, C. (Chairs), \emph{Educators meet the fifth estate:
Social media in education.} Symposium conducted at the American
Educational Research Association Annual Meeting, San Francisco, CA.
(Conference canceled)

\hangindent=2em Rosenberg, J. M., Hodge, L., Bertling, J., \& King, S.
(2020, April). Art as a context for data science: Exploring fourth-grade
students' data visualization practices. In J. M. Rosenberg \& B. Chen
(Chairs), \emph{Exploring data science across the curriculum and across
grade levels}. Symposium conducted at the American Educational Research
Association Annual Meeting, San Francisco, CA. (Conference canceled)

\hangindent=2em Rosenberg, J. M., Carpenter, J. P., Romero-Hall, E., \&
Kessler, A. (2020, April). \emph{Teacher educators' technology
competencies and the importance of context}. Paper presented at the
American Educational Research Association Annual Meeting, San Francisco,
CA. (Conference canceled)

\hangindent=2em Rosenberg, J. M., Beymer, P. N., Phun, V., Schmidt, J.
A. (2020, April). Sources of variability for students' engagement in
science: Findings from a cross-classified, multivariate modeling
approach. In P. N. Beymer, D. K. Benden, \& M. L. Bernacki (Chairs),
\emph{Affordances and modeling of intensive data}. Symposium conducted
at the American Educational Research Association Annual Meeting, San
Francisco, CA. (Conference canceled)

\hangindent=2em Rutherford, T., Rosenberg, J. M., \& Staudt Willet, K.
B. (2020, April). Which birds fill the branches of the AERA Twitter
tree? Twitter networks around \#AERA19. In P. N. Beymer, D. K. Benden,
\& M. L. Bernacki (Chairs), \emph{Affordances and modeling of intensive
data}. Symposium conducted at the American Educational Research
Association Annual Meeting, San Francisco, CA. (Conference canceled)

\hangindent=2em Jones, R. S., \& Rosenberg, J. M. (2020, April).
\emph{Latent class modeling of whole-class discussions about data,
statistics, and probability}. Paper presented at the American
Educational Research Association Annual Meeting, San Francisco, CA.
(Conference canceled).

\hangindent=2em Rutherford, T., \& Rosenberg, J. M. (2020, April).
\emph{Motivational correlates of choice to persist after failure}. Paper
presented at the American Educational Research Association Annual
Meeting, San Francisco, CA. (Conference canceled).

\hangindent=2em Ranellucci, J. \& Rosenberg, J. M. (2020, April).
\emph{Interest, engagement, and achievement in online high school
science courses}. Paper presented at the American Educational Research
Association Annual Meeting, San Francisco, CA. (Conference canceled).

\hangindent=2em Schmidt, J. A., Rosenberg, J. M., \& Beymer, P. N.
(August, 2019). \emph{Sources of variability in engagement: Exploring
situational, personal, and classroom influences}. Poster presented at
the annual meeting of the American Psychological Association, Chicago,
IL.

\hangindent=2em Greenhalgh, S. P., Huang, K., \& Rosenberg, J. M. (2019,
October). \emph{Understanding gaming communities and exploring learning
opportunities: A computational grounded theory approach}. Paper
presented at the meeting of the Association for Educational
Communications and Technology International Convention, Las Vegas, NV.

\hangindent=2em Rosenberg, J. M, Beymer, P. N., Houslay, T. M., \&
Schmidt, J. A. (2019, April). Using a multivariate, multi-level model to
understand how youths' in-the-moment engagement predicts changes in
youths' interest. In M. Bernacki, A. Kaplan, and L. Linnenbrink-Garcia
(Chairs), \emph{Embracing and modeling the complex dynamics of
motivation and engagement: Contextual, temporal, dynamic, and
systematic}. Symposium conducted at the Annual Meeting of the American
Educational Research Association, Toronto, CA.

\hangindent=2em Beymer, P. N., Schell, M. J., Alberts, K. M., Rosenberg,
J. M., \& Schmidt, J. A. (2019, April). \emph{Student engagement
profiles in formal and informal STEM learning settings}. Paper presented
at the Annual Meeting of the American Educational Research Association,
Toronto, Canada.

\hangindent=2em Schell, M. J., Beymer, P. N. Albert, K. M., Rosenberg,
J. M., \& Schmidt, J. A. (2019, April). \emph{Predictors of momentary
student engagement profiles in high school science classrooms}. Paper
presented at the Annual Meeting of the American Educational Research
Association, Toronto, Canada.

\hangindent=2em Reid, J., Rosenberg, J. M., Koehler, M. J., Fischer, C.,
\& McKenna, T. J. (2019, March). \emph{An exploration of \#NGSSchat
through social network analysis}. Paper presented at the National
Association for Research in Science Teaching Annual International
Conference, Baltimore, MD.

\hangindent=2em Rosenberg, J. M., Reid, J., Koehler, M. J., Fischer, C.,
\& McKenna, T. J. (2019, January). \emph{The roles of the Twitter
hashtag \#NGSSchat in the context of science education reform efforts}.
Paper presented at the Association for Science Teacher Education
International Meeting, Savannah, GA. (\emph{Nb. This paper was nominated
for the ASTE John C. Park National Technology Leadership Institute
Fellowship})

\hangindent=2em Akcaoglu, M., Hodges, C. B., Rosenberg, J. M., \&
Hilpert, J. (2018, October). \emph{Factors impacting middle school
students' interest, efficacy, and utility value of programming}. Paper
presented at the Association for Educational Communications and
Technology International Convention 2018. Kansas City, MO.

\hangindent=2em Staudt Willet, K. B., Greenhalgh, S. P., Rosenberg, J.
M., Koehler, M. J. (2018, October). \emph{Won't you be my neighbor? How
education stakeholders use hyperlinks to build information neighborhoods
on Twitter}. Paper presented at the Association for Educational
Communications and Technology International Convention 2018. Kansas
City, MO.

\hangindent=2em Beymer, P. N., Rosenberg, J. M., Schmidt, J. A.,
Naftzger, N. J., \& Shumow, L. (August, 2018). \emph{Agency in summer
STEM programs predicts interest and career aspirations}. Poster
presented at the annual meeting of the American Psychological
Association, San Francisco, CA.

\hangindent=2em Schmidt, J. A., Beymer, P. N., Rosenberg, J. M.,
Naftzger, N. J., \& Shumow, L. (August, 2018). \emph{Examining the
development of interest in summer STEM programs}. Poster presented at
the annual meeting of the American Psychological Association, San
Francisco, CA.

\hangindent=2em Beymer, P. N., Rosenberg, J.M., \& Schmidt, J. A. (2018,
April). \emph{Investigating the effects of interest and choice: An
experience sampling approach}. Paper presented at the Annual Meeting of
the American Educational Research Association, New York, NY.

\hangindent=2em Greenhalgh, S. P., Staudt Willet, B., Rosenberg, J. M.,
Akcaoglu, M., \& Koehler, M. J. (2018, April). \emph{Timing is
everything: Comparing synchronous and asynchronous modes of Twitter for
teacher professional learning}. Paper presented at the Annual Meeting of
the American Educational Research Association, New York, NY.

\hangindent=2em Rosenberg, J. M., Beymer, P. N., \& Schmidt, J. A.
(2018, April). \emph{How engagement during out-of-school time STEM
programs predicts changes in motivation in STEM}. In J. M. Rosenberg
(Chair), Data-intensive approaches to studying engagement in education:
Exploring their current potential. Paper presented at the Annual Meeting
of the American Educational Research Association, New York, NY.

\hangindent=2em Rosenberg, J. M., Lee, Y., Robinson, K. A., Ranellucci,
J., Roseth, C. J., \& Linnenbrink-Garcia, L. (2018, April).
\emph{Patterns of engagement in a flipped undergraduate class:
Antecedents and outcomes}. In L. Daniels \& A. Frenzel (Chairs), New
empirical insights on what energizes learners -- A session on emotions
and engagement. Paper presented at the Annual Meeting of the American
Educational Research Association, New York, NY.

\hangindent=2em Schmidt, J. A., Rosenberg, J.M., \& Beymer, P. N. (2018,
April). \emph{Experiences, activities, and personal characteristics as
predictors of interest and engagement in STEM-focused summer programs}.
Paper presented at the Annual Meeting of the American Educational
Research Association, New York, NY.

\hangindent=2em Shwartz, Y., Bayer, I., Bielik, T., Kolonich, A.,
Eidelman, R., Shwartz, G., . . . Rosenberg, J. M. (2018, March).
\emph{Graduate student international collaboration for investigating
science teachers' professional learning}. Paper presented at the meeting
of the National Association for Research in Science Teaching, Atlanta,
GA.

\hangindent=2em Yang, B. W., Blondel, D. V., Rosenberg, J. M., Sansone,
A., Linennbrink-Garcia, L., Schwarz-Bloom, R. D. (2017, November).
\emph{The Rex virtual experiment platform: Design, implementation, and
effects on situational interest}. Poster presented at the Annual Meeting
of the Society for Neuroscience, Washington, DC.

\hangindent=2em Greenhalgh, S. P., Staudt Willet, K. B., Rosenberg, J.
M., \& Koehler, M. J. (2017, November). \emph{No accounting for theory?
The case for an affinity space approach to educational hashtag
research}. Paper presented at the Association for Educational
Communications and Technology International Convention 2017,
Jacksonville, FL.

\hangindent=2em Greenhalgh, S. P., Rosenberg, J. M., \& Koehler, M. J.
(2017, November). \emph{Hide and go tweet: Comparing methods for
locating educational hashtag participants}. Paper presented at the
Association for Educational Communications and Technology International
Convention 2017, Jacksonville, FL.

\hangindent=2em Schmidt, J. A., Rosenberg, J. M., \& Beymer, P. N.
(2017, August). \emph{Stability and variation in student engagement in
science classes: A person-oriented approach}. Paper presented at the
Annual Meeting of the American Psychological Association, Washington,
DC.

\hangindent=2em Beymer, P. N., Rosenberg, J. M., Schmidt, J. A.,
Naftzger, N., Sniegowski, S., Shumow, L. (August, 2017). \emph{Examining
relationships between choice, affect, and engagement in informal STEM
programs}. Paper presented at the Annual Meeting of the American
Psychological Association, Washington, DC.

\hangindent=2em Greenhalgh, S. P., Rosenberg, J. M., \& Koehler, M. J.
(2017, April). \emph{Combining data sets and methods to explore equity
in teacher professional development. In D. G. Krutka (Chair), Data, big
and small}. Symposium conducted at the meeting of the American
Educational Research Association, San Antonio, TX.

\hangindent=2em Schmidt, J. A., Rosenberg, J. M., \& Beymer, P. N.
(2017, April). \emph{Momentary engagement profiles: A person-in-context
approach to studying student engagement using experience sampling data}.
Paper presented at the Annual Meeting of the American Educational
Research Association, San Antonio, TX.

\hangindent=2em Roseth, C. J., Linnenbrink-Garcia, L., Saltarelli, W.,
Lee, Y-K., Rosenberg, J. M. \ldots{} \& Beymer, P. N. (2017, April).
\emph{A design-based intervention on flipped instruction: Longitudinal
effects on undergraduates' engagement and achievement}. Paper presented
at the Annual Meeting of the American Educational Research Association,
San Antonio, TX.

\hangindent=2em Mikeska, J. N., Rosenberg, J. M., Holtzman, S., \&
McCaffrey, D. (2017, April). \emph{Comparing the alignment between two
observational measures of science teachers' instructional practice}.
Poster presented at the National Association for Research in Science
Teaching Annual International Conference, San Antonio, TX.

\hangindent=2em Greenhalgh, S. P., Rosenberg, J. M., \& Koehler, M. J.
(2017, March). \emph{Avoiding madness in our methods}. Paper presented
at the Society for Information Technology and Teacher Education
International Conference 2017, Austin, TX.

\hangindent=2em Rosenberg, J. M., Akcaoglu, M., Staudt Willet, K. B.,
Greenhalgh, S. P., \& Koehler, M. J. (2017, March). \emph{A tale of two
Twitters: Synchronous and asynchronous use of the same hashtag}. In P.
Resta \& S. Smith (Eds.), Proceedings of Society for Information
Technology \& Teacher Education International Conference 2017
(pp.~283-286). Waynesville, NC: Association for the Advancement of
Computing in Education (AACE).

\hangindent=2em Kessler, A., \& Rosenberg, J. M. (2017, March).
\emph{Considering how teachers' TPACK is leveraged during the mental
engineering of instruction: A theory of action}. Paper presented at the
Society for Information Technology and Teacher Education International
Conference 2017, Austin, TX.

\hangindent=2em Nyland, R., Greenhalgh, S. P., Rosenberg, J. M.,
Koehler, M. J., Veletsianos, G., \& Kimmons, R. (2016, October).
\emph{Public data mining methods, ethics, \& legalities}. Panel
presented at the Association for Educational Communications and
Technology International Convention 2016, Las Vegas, NV.

\hangindent=2em Rosenberg, J. M., Greenhalgh, S. P., \& Wolf, L. G.
(2016, October). \emph{Participating from near and far: Analyzing online
graduate learning communities with social network analysis}. Paper
presented at the Association for Educational Communications and
Technology International Convention 2016, Las Vegas, NV.

\hangindent=2em Rosenberg, J. M. (2016, October). \emph{Having agency in
conditions that are not equitable: An examination of Donors Choose
data}. Paper presented at the Association for Educational Communications
and Technology International Convention 2016, Las Vegas, NV.

\hangindent=2em Phillips, M., Koehler, M. J., \& Rosenberg, J. M. (2016,
September). \emph{Contextualising teachers' TPACK development and
enactment}. Paper presented at the Australian Council for Computers in
Education, Brisbane, Australia.

\hangindent=2em Rosenberg, J. M. \& Schwarz, C. V. (2016, April).
Examining fifth and sixth grade students' epistemic considerations
through an automated analysis of embedded assessment items. In B. Reiser
(Chair), \emph{Longitudinal studies of elementary and middle school
students' epistemic considerations through participation in scientific
practice}. Related paper set presented at the National Association for
Research in Science Teaching Annual International Conference, Baltimore,
MD. (slides)

\hangindent=2em Rosenberg, J. M. \& Krist, C. (2016, April).
\emph{Characterizing students' epistemic considerations: An automated
computational approach for embedded assessment responses}. Poster
presented at the National Association for Research in Science Teaching
Annual International Conference, Baltimore, MD. (slides)

\hangindent=2em Ranellucci, J., Rosenberg, J. M., Klautke, H., Robinson,
K. A., Saltarelli, W., Linnenbrink-Garcia, L., \& Roseth, C. J. (2016,
April). \emph{Achievement goals, behavioral engagement, and achievement
in a flipped undergraduate anatomy course}. Paper presented at the
Annual Meeting of the American Educational Research Association,
Washington, DC.

\hangindent=2em Lee, Y.-K., Rosenberg, J. M., Robinson, K. A., Klautke,
H., Seals, C., Saltarelli, W., Linnenbrink-Garcia, L., \& Roseth, C. J.
(2016, April). \emph{Comparing motivation and achievement in a flipped
and traditional classroom}. Paper presented at the Annual Meeting of the
American Educational Research Association, Washington, DC.

\hangindent=2em Wormington, S. V., Lee, Y.-K., Seals, C., Rosenberg, J.
M., Saltarelli, W., Roseth, C. J., \& Linnenbrink-Garcia, L. (2016,
April). \emph{Predicting profile permanence: When is motivation stable,
why does it change, and what are the consequences?} Paper presented at
the Annual Meeting of the American Educational Research Association,
Washington, DC.

\hangindent=2em Ranellucci, J., Robinson, K. A., Rosenberg, J. M.,
Saltarelli, W., Roseth, C. J., \& Linnenbrink-Garcia, L. (2016, April).
\emph{Comparing emotions in-class and during online video lectures in a
flipped classroom}. Paper presented at the Annual Meeting of the
American Educational Research Association, Washington, DC.

\hangindent=2em Rosenberg, J. M., Ranellucci, J., Lee, Y.-K., Robinson,
K., Saltarelli, W., Linnenbrink-Garcia, L., \& Roseth, C. J. (2016,
March). \emph{Patterns of engagement in a flipped undergraduate anatomy
class and their relations to achievement}. Paper presented at the
Society for Information Technology \& Teacher Education Annual
Conference, Savannah, GA.

\hangindent=2em Rosenberg, J. M. (2015, November). \emph{Examining what
teachers and researchers discuss at science education conferences
through a computational analysis of Twitter data}. Paper presented at
the meeting of the Association for Educational Communications and
Technology, Indianapolis, IN.

\hangindent=2em Rosenberg, J. M., Akcaoglu, M., Hamilton, E.,
Greenhalgh, S. P., \& Koehler, M. J. (2015, November). \emph{Tweeting
U.S.A.: An examination of State Educational Twitter Hashtags (SETHs)}.
Paper presented at the meeting of the Association for Educational
Communications and Technology, Indianapolis, IN.

\hangindent=2em Greenhalgh, S. P., Rosenberg, J. M., Keenan, S., \&
Koehler, M. J. (2015, November). \emph{An investigation of the use of
digital portfolios for understanding educators' technology knowledge}.
Paper presented at the meeting of the Association for Educational
Communications and Technology, Indianapolis, IN.

\hangindent=2em Hamilton, E., Rosenberg, J. M., \& Akcaoglu, M. (2015,
November). \emph{Examining the Substitution Augmentation Modification
Redefinition (SAMR) Model for instructional design and technology
integration}. Paper presented at the meeting of the Association for
Educational Communications and Technology, Indianapolis, IN.

\hangindent=2em Mehta, R., Rosenberg, J. M., Russo, T., Arnold, B.,
Marich, H., \& Bell, J. (2015, November). \emph{A survey of social media
use and the effects of a social media initiative on graduate student
engagement}. Paper presented at the meeting of the Association for
Educational Communications and Technology, Indianapolis, IN.

\hangindent=2em Rosenberg, J. M., \& Koehler, M. J. (2015, April).
Context and Technological Pedagogical Content Knowledge: A content
analysis. In J. M. Rosenberg \& M. J. Koehler (Chairs), \emph{Addressing
the complexity of teaching with technology: Context and Technological
Pedagogical Content Knowledge}. Symposium conducted at the American
Educational Research Association Annual Meeting, Chicago, IL.

\hangindent=2em Hamilton, E., Rosenberg, J. M., \& Akcaoglu, M. (2015,
April). \emph{The Substitution Augmentation Modification Redefinition
(SAMR) framework for technology integration: Challenges to its use for
guiding K-12 teacher's pedagogy and practice}. Paper presented at the
American Educational Research Association Annual Meeting, Chicago, IL.

\hangindent=2em Rosenberg, J. M., Ervin, L., Harris, J., Greenhow, C.,
Kessler, A., \& Tai, D. (2015, March). \emph{How should educational
technology researchers consider context? An interactive discussion on
context and teaching and learning with technology}. Panel presented at
the meeting of the Society for Information Technology and Teacher
Education International Conference, Las Vegas, NV.

\hangindent=2em Akcaoglu, M., \& Rosenberg, J. M. (2015, March).
\emph{Best practices for designing synchronous and asynchronous online
teaching for adult learners}. Poster presented at the meeting of the
Society for Information Technology and Teacher Education, Las Vegas, NV.

\hangindent=2em Rosenberg, J. M., Schwarz, C. V., \& Lee, S. W.-Y., \&
Akcaoglu, M. (2015, April). A comparative longitudinal case study of the
use of scientific modeling in the pedagogical practice of two
fifth-grade science teachers. In A. Lo (Chair), \emph{Leveraging the
epistemic dimensions of scientific practice to support students'
meaningful engagement in modeling}. Related paper set presented at the
National Association for Research in Science Teaching Annual
International Conference, Chicago, IL.

\hangindent=2em Rosenberg, J. M., Schwarz, C.V., Akcaoglu, M., \& Lee,
S.W-Y. (2014, October). \emph{Comparative longitudinal case studies of
two middle school teachers' use of scientific modeling}. Poster
presented at the Advances in Educational Psychology Conference. Fairfax,
VA.

\hangindent=2em Lee, M., Schwarz, C. V., Ke, L., \& Rosenberg, J. M.
(2014, April). \emph{Analyzing fifth-grade students' engagement in
scientific modeling: Changes in students' epistemologies-in-practice
over time}. Paper presented at the meeting of the National Association
for Research in Science Teaching, Philadelphia, PA.

\hangindent=2em Ke, L., Schwarz, C. V., Lee, M. \& Rosenberg, J. M.
(2014, April). \emph{Examining elementary students' attention to
mechanism as they engage in scientific modeling across content areas}.
Paper presented at the meeting of the National Association for Research
in Science Teaching, Philadelphia, PA.

\hangindent=2em Koehler, M. J., Rosenberg, J. M., Greenhalgh, S.,
Zellner, A. L., \& Mishra, P. (2014, March). Analyzing students'
portfolios for the development of TPACK. In J. Voogt (Chair),
\emph{Artifacts demonstrating teachers' technology integration
competencies}. Symposium presented at the meeting of the Society for
Information Technology and Teacher Education, Jacksonville, FL.

\hypertarget{invited-talks}{%
\subsection{Invited Talks}\label{invited-talks}}

\hangindent=2em Rosenberg, J. M. (February, 2020). \emph{Studying
education-focused Twitter hashtags in light of state-based and national
policies and practices}. Presentation at the 2020 Spring Seminar Series
at the Martin School of Public Policy at the University of Kentucky,
Lexington, KY.

\hangindent=2em Rosenberg, J.M. (September, 2019). \emph{Making data
science education count: Exploring the integration of data science into
education}. Presentation at the Middle Tennessee State University
Mathematics and Science Education Doctoral Seminar series. Middle
Tennessee State University, Murfreesboro, TN.

\hangindent=2em Rosenberg, J. M. (February, 2019). \emph{Making sense of
recent advances in the Technological Pedagogical Content Knowledge
framework}. English International Congress at the Universidad Técnica
del Norte, Ibarra, Ecuador.

\hypertarget{other-presentations}{%
\subsection{Other Presentations}\label{other-presentations}}

\hangindent=2em Rosenberg, J. M., Dyer, E. B., Anderson, D. J., \&
Fischer, C. (September, 2020). \emph{If you're happy and you know it,
post a tweet? A study of the sentiment of posts to the \#NGSSchat
hashtag on Twitter}. Presentation at the AERA Satellite Conference on
Educational Data Science, Stanford, CA.

\hangindent=2em Dyer, E. B., Rosenberg, J. M., Bosch, N., Krist, C., \&
D'Angelo, C. (September, 2020). \emph{Better together? Initial findings
and implications from combining qualitative coding and computational
methods to analyze classroom audiovisual data}. Presentation at the AERA
Satellite Conference on Educational Data Science, Stanford, CA.

\hangindent=2em Anderson, D., Rosenberg, J. M., Sáez, L., \& Seeley, J.
R. (September, 2020). \emph{Using extreme gradient boosting to estimate
community effects on school readiness}. Presentation at the AERA
Satellite Conference on Educational Data Science, Stanford, CA.

\hangindent=2em Estrellado, R. A., Bovee, E. A., Mostipak, J.,
Rosenberg, J. M., \& Velásquez, I. C. (July, 2020). \emph{Expanding R
into education}. Presentation at the useR conference, St.~Louis, MO.

\hangindent=2em Rosenberg, J. M., Qinyun, L., Xu, R., Maroulis, S., \&
Frank, K. A. (July, 2020). \emph{The konfound R package and Shiny app
for robustness analysis}. Presentation at the useR conference,
St.~Louis, MO.

\hangindent=2em Rosenberg, J. M.,\& Lishinski, A. (January, 2020).
\emph{Measuring what matters in-the-moment: An experience sampling
approach to understanding the development of interest in computer
science}. Presentation at the 14th Annual Tennessee STEM Education
Research Conference, Cookeville, TN.

\hangindent=2em Rosenberg, J. M., Hodge, L., Aydeniz, M., Schmidt, A.
Lishinski, A., Rich, K., Longnecker, J., Mann. M., \& Sadovnik, A.
(January, 2020). \emph{A survey of teachers and administrators regarding
the implementation of new K-8 computing education standards in
Tennessee.} Presentation at the 14th Annual Tennessee STEM Education
Research Conference, Cookeville, TN.

\hangindent=2em Camponovo, M., Lawson, M. A., \& Rosenberg, M. J. (July,
2019). \emph{Integrating geospatial tech with math and science
pre-service teachers. 2019 Education Summit @ ESRI UC}. San Diego, CA.

\hangindent=2em Jones, R. S., \& Rosenberg, J. M. (February, 2019).
\emph{Latent class modeling of whole class discussions about data,
statistics, and probability. Presentation at the 13th Annual Tennessee
STEM Education Research Conference}, Murfreesboro, TN.

\hangindent=2em Lawson, M., Rosenberg, J. M., \& Camponovo, M.
(February, 2019). \emph{Better together? Findings from a combined,
integrated STEM unit with pre-service mathematics and science teachers}.
Presentation at the 13th Annual Tennessee STEM Education Research
Conference, Murfreesboro, TN.

\hypertarget{workshops}{%
\subsection{Workshops}\label{workshops}}

\hangindent=2em Sorge, S., Kubsch, M., Rosenberg, J. M., \& D'Angelo, C.
(2021, April). \emph{Rethinking how you understand your data with R}.
Workshop carried out at the National Association for Research in Science
Teaching.

\hangindent=2em Dyer, E. B., D'Angelo, D., Bosch, N., Krist, C., \&
Rosenberg, J. M. (2020, June). \emph{Analyzing learning with speech
analytics and computer vision methods: Technologies, principles, and
ethics}. Workshop carried out at the International Conference of the
Learning Sciences, Nashville, TN.

\hangindent=2em Staudt Willet, K. B., Rosenberg, J. M., \& Greenhalgh,
S. P. (2020, March). \emph{R U ready 4 R? Introduction to Analyzing
Educational Internet Data Using R}. Workshop carried out for the
\emph{Students, Social Media, and Schools Research Group} at Florida
State University, Talahassee, FL.

\hangindent=2em Rosenberg, J. M. (2020, January). \emph{An introduction
to using R for data science (zero prerequisites required!)}. Workshop
carried out for the KnoxData group, Knoxville, TN.

\hangindent=2em Rosenberg, J. M., Staudt Willet, K. B., \& Greenhalgh,
S. P. (2019, October). \emph{Online data and open source tools:
Analyzing educational internet data Using R}. Workshop carried out at
the Association for Educational Communications and Technology, Las
Vegas, NV.

\hangindent=2em Rosenberg, J.M. (September, 2019). \emph{An introduction
to data science in education using R}. Workshop at Middle Tennessee
State University. Middle Tennessee State University, Murfreesboro, TN.

\hangindent=2em Rosenberg, J. M. (2019, June). \emph{The use of mixed
effects models for analyzing complex data}. Presentation for the
KnoxData group, Knoxville, TN. YouTube recording:
\url{https://www.youtube.com/watch?v=1YY2FoCFIm4}

\hangindent=2em Rosenberg, J. M. (2019, May). \emph{Won't you be my
neighboR? An introduction to R for data science in education}. Workshop
carried out for the Educational Psychology and Educational Technology
program, Michigan State University.

\hangindent=2em Anderson, D. J., and Rosenberg, J. M. (2019, April).
\emph{Transparent and reproducible research with R}. Workshop carried
out at the Annual Meeting of the American Educational Research
Association, Toronto, Canada.

\hangindent=2em Rosenberg, J. M. (2017, April). \emph{Introduction to R
for Data Analysis}. Presentation at the School of Criminal Justice,
Michigan State University.

\hangindent=2em Rosenberg, J. M. (March, 2016). \emph{An introduction to
R for programming and statistical analysis in education}. Georgia
Southern University College of Education, Statesboro, GA.

\hypertarget{outreach-and-engagement}{%
\section{Outreach and Engagement}\label{outreach-and-engagement}}

\hangindent=2em Rosenberg, J. M. (2020, April). \emph{An informal, open
introduction to using R Markdown in education}. Virtual workshop.
\url{https://www.youtube.com/watch?v=BA1YFvmXCXQ\&t=57s} s
\hangindent=2em Rosenberg, J. M. (2019, May). \emph{Working with data in
education: Using data and supporting students to use data}. Workshop
carried out for teachers at Knoxville Jewish Day School.
\url{https://docs.google.com/presentation/d/1uSdRvF2GjhUpO2fCHZIUdXmf0texzczGGlbzmZBgggw/edit?usp=sharing}

\hangindent=2em Trout-Fryxell, B., \& Rosenberg, J. M. (2020, February).
\emph{Authentic science in the classroom with MEGA:BITESS}. Presentation
at the Knox County Schools Science Department District Learning Day,
Knoxville, TN.

\hangindent=2em Ranellucci, J., \& Rosenberg, J. M. (2016, February).
\emph{Motivating our students: A partnership between Michigan Virtual
Schools and Michigan State University}. Workshop at Michigan Virtual
University, East Lansing, MI.

\hangindent=2em Rosenberg, J. M. (2014, April). \emph{Action research
with mobile devices and other ``disruptive'' technologies}. Presentation
at the Best of the Michigan Association for Computer Users in Learning
Conference, Waterford, MI.

\hangindent=2em Rosenberg, J. M. (2014, February). \emph{Action research
with mobile devices}. Presentation at the Michigan Association for
Computer Users in Learning Mobile Learning Conference, Kalamazoo, MI.

\hangindent=2em Sawaya, S., \& Rosenberg, J. M (2014, February).
\emph{Master of Arts in Educational Technology Mobile Learning
Workshop}. Workshop at Michigan State University, East Lansing, MI.

\hypertarget{teaching}{%
\section{Teaching}\label{teaching}}

\hypertarget{teaching-awards}{%
\subsubsection{Teaching Awards}\label{teaching-awards}}

MSU-AT\&T Instructional Technology Award: Best Online Course, 2014

MSU-AT\&T Instructional Technology Award (Honorable Mention): Best
Online Course, 2013

\hypertarget{courses-taught}{%
\subsubsection{Courses Taught}\label{courses-taught}}

Instructor at the University of Tennessee, Knoxville:

\emph{Introduction to Data Science Methods in Education} (TPTE 595 \&
TPTE 695, M.A.~and Ph.D.~class) \emph{Nature of Mathematics and Science
Education} (SCED 572, M.A.~and Ph.D.~class)\\
\emph{Teaching Science in Grades 7-12} (TPTE 495, SCED 496, \& SCED 543,
B.S. \& M.A.~class)\\
\emph{VolsTeach Step 1 and Step 2} (TPTE 110 \& TPTE 120,
undergraduate-level class)

Instructor at Michigan State University:

\emph{Psychology of Learning in School and Other Settings} (CEP 800,
M.A.~class)\\
\emph{Approaches to Educational Research} (CEP 822, M.A.~class)\\
\emph{Technology and Leadership} (CEP 815, M.A.~class)

Teaching Assistant at Michigan State University:

\emph{Proseminar in Educational Psychology and Educational Technology}
(CEP 900, Ph.D.~class)\\
\emph{Proseminar in Educational Technology} (CEP 807 / ED 870,
M.A.~class)\\
\emph{Educational Inquiry} (CEP 900, Ph.D.~class)\\
\emph{Social-Emotional Development Across the Lifespan} (CEP 904,
Ph.D.~class)

\hypertarget{service}{%
\section{Service}\label{service}}

\hypertarget{editorial-service}{%
\subsubsection{Editorial Service}\label{editorial-service}}

Editorial Review Board Member, \emph{Journal of Research in Science
Teaching}, 2019-2022

Editorial Review Board Member, \emph{Contemporary Issues in Technology
and Teacher Education (Science Education Section)}, 2019 - Present

Editorial Review Board Member, \emph{Journal of Research on Technology
in Education}, 2016 - Present

Special Issue Editor, \emph{Australasian Journal of Educational
Technology}, 2017

\hypertarget{service-to-the-profession}{%
\subsubsection{Service to the
Profession}\label{service-to-the-profession}}

Panelist, Building Capacity in STEM Education Research, National Science
Foundation, \emph{n.d.}

Panelist, Discovery Research PreK-12, National Science Foundation,
\emph{n.d.}

Panelist, Innovative Technology Experiences for Students and Teachers,
National Science Foundation, \emph{n.d.}

American Educational Research Association, Division C, Section 1D:
Science Program Co-Chair, 2019-2021

Member, Technological Pedagogical Content Knowledge (TPACK) Special
Interest Group (SIG) Award Committee, 2019

Co-chair, TPACK SIG, Society for Information Technology and Teacher
Education , 2015-2017

Membership Committee, Division 15 (Educational Psychology), American
Psychological Association (APA), 2014-2017

Communications Deputy, Division C, American Educational Research
Association, 2015-2016

Associate Chair, TPACK SIG, Society for Information Technology and
Teacher Education, 2014-2015

\hypertarget{conference-review-activity}{%
\subsubsection{Conference Review
Activity}\label{conference-review-activity}}

Review Panel Member, American Educational Research Association (AERA)
Annual Meeting, 2015-2019

Reviewer, National Association for Research in Science Teaching Annual
Conference, 2019

Reviewer, Association for Science Teacher Education Annual Conference,
2019

Program Committee Member, International Conference on Computer-Supported
Collaborative Learning, 2017

Graduate Student Reviewer, American Educational Research Association
(AERA) Annual Meeting, 2014

Reviewer, Association for Educational Communications and Technology
(AECT) International Convention, 2016

Reviewer, American Psychological Association (APA) Convention, 2015

\hypertarget{service-to-the-community}{%
\subsubsection{Service to the
Community}\label{service-to-the-community}}

Mentor, {[}Diversity in Learning Analytics and Leadership program

Reviewer, Proposals from Knox County Schools students for the NASA
Student Spaceflight Experiment program

\hypertarget{ad-hoc-journal-article-reviews}{%
\subsubsection{Ad-hoc Journal Article
Reviews}\label{ad-hoc-journal-article-reviews}}

AERA Open (2019, 2020)\\
Australasian Journal of Educational Technology (2018: 2)\\
British Journal of Educational Technology (2016)\\
Computers \& Education (2016, 2017, 2018, 2020)\\
Contemporary Educational Psychology (2018)\\
Contemporary Issues in Technology and Teacher Education (2015)\\
Educational Researcher (2020)\\
Education Sciences (2; 2019)\\
Educational Studies in Mathematics (2020) Educational Technology
Research \& Development (2020) E-Learning and Digital Media (2016: 2)\\
Journal of Educational Technology \& Society (2017)\\
Journal of the Learning Sciences (2019)\\
Journal of Open Source Education (2019)\\
Journal of Open Source Software (2018; 2020)\\
Journal of Research in Science Teaching (2019)\\
Journal of Science Education and Technology (2019: 2; 2020)\\
Journal of STEM Education Research (2019)\\
Science Education (2021) TechTrends (2019)

\hypertarget{college-related-service}{%
\subsubsection{College-related Service}\label{college-related-service}}

Facilitator, Quality Research and Scholarship working group, 2020

Member, Online Academic Programs Investment and Growth Plan ad-hoc
committee

Organizer, Quantitative Methods Research Group

\hypertarget{departmental-service}{%
\subsubsection{Departmental Service}\label{departmental-service}}

\emph{University of Tennessee, Knoxville}

Member, Annual Review Rubrics Committee

Mentor, AERA Bootcamp (2019)

\emph{Michigan State University}

Search Committee Member, Program Specialist, Master of Arts in
Educational Technology Program, Michigan State University, 2015

\hypertarget{program-service-and-service-on-student-committees}{%
\subsubsection{Program Service and Service on Student
Committees}\label{program-service-and-service-on-student-committees}}

\emph{University of Tennessee, Knoxville}

Advisor for Doctoral students:

Jennifer Longnecker (Co-advisor with Amy Broemmel)\\
Michael Mann (co-advisor with Kristin Rearden)\\
Omiya Sultana (co-advisor with Lynn Hodge)

Committee member for Doctoral students:

Shande King\\
Matthew Hensley

\emph{Michigan State University}

Member of two practica committees for Educational Psychology and
Educational Technology program Ph.D.~students, Michigan State
University, 2014-2018

\hypertarget{campus-and-departmental-presentations}{%
\subsubsection{Campus and Departmental
Presentations}\label{campus-and-departmental-presentations}}

\hangindent=2em Rosenberg, J. M. (2020, January). \emph{Multiple uses
for multi-level models: Examples from recent research}. Presentation for
the College of Education, Health, and Human Sciences Quantitative
Methods Brownbag Seminar, Knoxville, TN.

\hangindent=2em Rutherford, T., \& Rosenberg, J. M. (2019, February).
\emph{Motivational correlates of choice after failure within an
elementary mathematics software}. Presentation at the NC State College
of Education Celebration of Research.

\hangindent=2em Rosenberg, J. M. (2019, January). \emph{Engaging
students in science: Findings from an experience sampling method
approach}. Presentation at the East Tennessee STEM Hub Crossing
Boundaries for STEM Teaching regional meeting and mini-conference.
Knoxville, TN.

\hangindent=2em Rosenberg, J. M., Beymer, P. N., \& Schmidt, J. A.
(2017, February). \emph{Does choosing the problem or topic matter? Using
a person-in-context approach to understand student engagement in
science}. Poster presented at the Create4Stem MiniConference 2017, East
Lansing, MI.

\hangindent=2em Rosenberg, J. M. (2016, April). \emph{Momentary
engagement profiles: An examination of student engagement in science
settings using experience sampling methodology}. Presentation at the
Michigan State University Educational Psychology and Educational
Technology Program Informal Colloquium, East Lansing, MI.

\hangindent=2em Rosenberg, J. M., \& Schwarz, C. V. (2016, February).
\emph{Examining the development of fifth and sixth grade students'
epistemic considerations over time through an automated analysis of
embedded assessment items}. Poster presented at the Create4Stem
MiniConference 2016, East Lansing, MI.

\hangindent=2em Rosenberg, J. M. (2015, September). \emph{Achievement
goals, in- and out-of-class engagement, and students' achievement in a
flipped undergraduate anatomy class}. Presentation at the Michigan State
University Educational Psychology and Educational Technology Program
Informal Colloquium, East Lansing, MI.

\hangindent=2em Rosenberg, J. M., Akcaoglu, M., Schwarz, C.V., \& Lee,
S.W-Y. (2015, February). \emph{Comparative longitudinal case studies of
two middle school teachers' use of scientific modeling}. Poster
presented at the Create4Stem MiniConference 2015, East Lansing, MI.

\hangindent=2em Lee, M., Schwarz, C.V., Ke, L., Rosenberg, J. M.,
Reiser, B., Berland, L., Kenyon, L., Wilson, M., Draney, K. (2015,
February). \emph{Epistemic considerations in scientific practices for
elementary \& middle schools}. Poster presented at the Create4Stem
MiniConference 2015, East Lansing, MI.

\hangindent=2em Wolf, L. G., Henriksen, D., Sawaya, S., \& Rosenberg, J.
M. (2014, December). \emph{EdCamp with Team MAET}. Presentation at the
Michigan State University Master of Arts in Educational Technology
Bridge Webinar Series, East Lansing, MI.

\hangindent=2em Rosenberg, J. M. (2014, November). \emph{Integrating
``disruptive'' technologies into teaching with action research and
Technological Pedagogical Content Knowledge (TPACK)}. Presentation at
the Michigan State University Educational Technology Conference, East
Lansing, MI.

\hangindent=2em Wolf, L. G., Henriksen, D., Sawaya, S., \& Rosenberg, J.
M. (2014, March). \emph{Mobile learning for educators}. Presentation at
the Michigan State University Master of Arts in Educational Technology
Bridge Webinar Series, East Lansing, MI.

\hangindent=2em Rosenberg, J. M. (2014, February). \emph{Context and
Technological Pedagogical Content Knowledge: Preliminary results of a
content analysis}. Presentation at the Michigan State University
Educational Psychology and Educational Technology Program Informal
Colloquium, East Lansing, MI.

\hangindent=2em Ke, L., Lee, M., Rosenberg, J. M., \& Schwarz, C.V.
(2014, February). \emph{Modeling across content areas: Examining
elementary students' attention to mechanism}. Poster presented at the
Create4Stem MiniConference 2014, East Lansing, MI.

\hangindent=2em Rosenberg, J. M., Rapa, L., \& Wolf, L. G. (2013,
February). \emph{CEP 815 and the transition from ANGEL to Desire2Learn}.
Poster presented at the 6th Annual Faculty Technology Showcase.

\hangindent=2em Rosenberg, J. M. (2012, November). \emph{Mobile learning
for teachers}. Presentation at the Michigan State University Educational
Technology Conference, East Lansing, MI.

\hypertarget{software}{%
\section{Software}\label{software}}

\hypertarget{author-of-r-packages-on-comprehensive-r-archive-network-cran}{%
\subsubsection{Author of R packages on Comprehensive R Archive Network
(CRAN)}\label{author-of-r-packages-on-comprehensive-r-archive-network-cran}}

Rosenberg, J. M., van Lissa, C. J., Beymer, P. N., Anderson, D. J.,
Schell, M. J. \& Schmidt, J. A. (2019). \emph{tidyLPA: Easily carry out
Latent Profile Analysis (LPA) using open-source or commercial software}
{[}R package{]}. \url{https://data-edu.github.io/tidyLPA/}

Rosenberg, J. M., Xu, R., \& Frank, K. A. (2019). \emph{konfound:
Quantify the robustness of causal inferences} {[}R package{]}.
\url{https://jrosen48.github.io/konfound/}

Rosenberg, J. M., Schmidt, J. A., Beymer, P. N., \& Steingut, R. (2018).
\emph{prcr: Person-Centered Analysis} {[}R package{]}.
\url{https://CRAN.R-project.org/package=prcr}

Rosenberg, J. M., \& Lishinski, A. (2018). \emph{clustRcompaR: Easy
interface for clustering a set of documents and exploring group-based
patterns} {[}R package{]}.
\url{https://github.com/alishinski/clustRcompaR}

\hypertarget{contributor-to-r-package-on-cran}{%
\subsubsection{Contributor to R package on
CRAN}\label{contributor-to-r-package-on-cran}}

D'Agostina McGowan, L., Hester, J., Rosenberg, J. M., \& Leek, J.
(2020). \emph{tidycode: Analyze Lines of R Code the Tidy Way}.
\url{https://github.com/LucyMcGowan/tidycode}

\hypertarget{author-of-r-packages-on-github}{%
\subsubsection{Author of R packages on
GitHub}\label{author-of-r-packages-on-github}}

Estrellado, R. A., Bovee, E. A., Mostipak, J., Rosenberg, J. M., \&
Velásquez, I. C. (2019). \emph{dataedu: Package for Data Science in
Education Using R}. \url{https://github.com/data-edu/dataedu}

Anderson, D. Heiss, A., and Rosenberg, J. M. (2019). \emph{equatiomatic:
Transform Models into LaTeX Equations.}
\url{https://github.com/datalorax/equatiomatic}

Velásquez, I. and Rosenberg, J. M. (2019). \emph{leaidr: U.S. School
District Shapefiles} \url{https://github.com/ivelasq/leaidr}

Seo, J., \& Rosenberg, J. M. (2020). \emph{jladown: Writing a
Reproducible Article for Journal of Learning Analytics in R Markdown}.
\url{https://github.com/jooyoungseo/jladown}

Staudt Willet, B., \& Rosenberg, J. M. (2020). \emph{tidytags: Simple
Collection and Powerful Analysis of Twitter Data}
\url{https://github.com/bretsw/tidytags}

Rosenberg, J. M. (2020). \emph{tidykids: State-by-State Spending on Kids
Dataset}. \url{https://jrosen48.github.io/tidykids/}

\hypertarget{interactive-web-applications}{%
\subsubsection{Interactive Web
Applications}\label{interactive-web-applications}}

Rosenberg, J. M., Xu, R., \& Frank, K. A. (2019). \emph{Konfound-It!:
Quantify the robustness of causal inferences.}
\url{http://konfound-it.com}.

Rosenberg, J. M., \& Krist, C. (2019). \emph{Generality embedded
assessment classifier.}
\url{https://jmichaelrosenberg.shinyapps.io/generality-shiny/}

Rosenberg, J. M. (2019). \emph{How many (MCMC) cores?}
\url{https://jmichaelrosenberg.shinyapps.io/how-many-cores/}

Rosenberg, J. M. (2016). \emph{State Educational Twitter Hashtags
(SETHs).} \url{https://jmichaelrosenberg.shinyapps.io/SETHs/}

\hypertarget{computational-science-simulation}{%
\subsubsection{Computational Science
Simulation}\label{computational-science-simulation}}

Rosenberg, J. M. (2016). Diffusion \& temperature. Lab Interactive
Simulation.
\url{https://lab.concord.org/interactives.html\#interactives/external-projects/msu/temperature-diffusion.json}

\hypertarget{python-based-web-application}{%
\subsubsection{Python-Based Web
Application}\label{python-based-web-application}}

Lishinski, A., \& Rosenberg, J. M. (2019). \emph{Short message survey:
An open-source, text-message based application for the experience
sampling method.} \url{https://github.com/picsul/short-message-survey}

\hypertarget{other-projects}{%
\subsubsection{Other Projects}\label{other-projects}}

I have contributed to a number of open-source projects by filing issues
or making suggestions:
\url{https://github.com/search?q=is\%3Aissue+author\%3Ajrosen48\&type=Issues}

\hypertarget{miscellaneous}{%
\section{Miscellaneous}\label{miscellaneous}}

\hypertarget{competitive-research-training}{%
\subsubsection{Competitive Research
Training}\label{competitive-research-training}}

Early Career Workshop, International Conference of the Learning
Sciences, 2020

New Faculty Mentoring Program, AERA Division C, 2019

Graduate Student Seminar, AERA Division C, 2016

Early Career Seminar, Association for Educational Communications and
Technology, 2015

Research Methods with Diverse Groups Advanced Training Institute,
American Psychological Association, 2014

\hypertarget{media}{%
\subsubsection{Media}\label{media}}

2020, Education Data Chat podcast,
\url{https://www.buzzsprout.com/1074286/4993430}

2019, Flowing Data, Teaching R to 7th Graders,
\url{https://flowingdata.com/2019/11/26/teaching-r-to-7th-graders/}

2016, Innovative Education in VT,
\url{https://tiie.w3.uvm.edu/blog/educators-on-twitter/\#.XzkFq5NKiHE}

\hypertarget{podcast}{%
\subsubsection{Podcast}\label{podcast}}

2018-2019, Co-host,
\href{http://impodstersyndrome.libsyn.com/}{\emph{Impodster Syndrome}
podcast}

\hypertarget{consulting}{%
\subsubsection{Consulting}\label{consulting}}

2017-2020, Senior Investigating Consultant, \emph{Profiles of science
engagement: Broadening participation by understanding individual and
contextual influences}, Jennifer Schmidt, Michigan State University.
(NSF Grant
No.~\href{https://nsf.gov/awardsearch/showAward?AWD_ID=1661064\&HistoricalAwards=false}{1661064})

2017-2019, Statistical software development, Kenneth Frank, Michigan
State University

2017, Statistical analysis, Yael Shwartz, Weizmann Institute

2016, Statistical analysis, Lara Kassab, San Jose State University

\hypertarget{professional-affiliations}{%
\subsubsection{Professional
Affiliations}\label{professional-affiliations}}

American Educational Research Association, 2012 - Present\\
International Society of the Learning Sciences, 2014 - Present\\
National Association for Research in Science Teaching, 2015 - Present

\end{document}

% Options for packages loaded elsewhere
\PassOptionsToPackage{unicode}{hyperref}
\PassOptionsToPackage{hyphens}{url}
%
\documentclass[
  14,
]{article}
\usepackage{amsmath,amssymb}
\usepackage{iftex}
\ifPDFTeX
  \usepackage[T1]{fontenc}
  \usepackage[utf8]{inputenc}
  \usepackage{textcomp} % provide euro and other symbols
\else % if luatex or xetex
  \usepackage{unicode-math} % this also loads fontspec
  \defaultfontfeatures{Scale=MatchLowercase}
  \defaultfontfeatures[\rmfamily]{Ligatures=TeX,Scale=1}
\fi
\usepackage{lmodern}
\ifPDFTeX\else
  % xetex/luatex font selection
\fi
% Use upquote if available, for straight quotes in verbatim environments
\IfFileExists{upquote.sty}{\usepackage{upquote}}{}
\IfFileExists{microtype.sty}{% use microtype if available
  \usepackage[]{microtype}
  \UseMicrotypeSet[protrusion]{basicmath} % disable protrusion for tt fonts
}{}
\makeatletter
\@ifundefined{KOMAClassName}{% if non-KOMA class
  \IfFileExists{parskip.sty}{%
    \usepackage{parskip}
  }{% else
    \setlength{\parindent}{0pt}
    \setlength{\parskip}{6pt plus 2pt minus 1pt}}
}{% if KOMA class
  \KOMAoptions{parskip=half}}
\makeatother
\usepackage{xcolor}
\usepackage[margin=1in]{geometry}
\usepackage{graphicx}
\makeatletter
\def\maxwidth{\ifdim\Gin@nat@width>\linewidth\linewidth\else\Gin@nat@width\fi}
\def\maxheight{\ifdim\Gin@nat@height>\textheight\textheight\else\Gin@nat@height\fi}
\makeatother
% Scale images if necessary, so that they will not overflow the page
% margins by default, and it is still possible to overwrite the defaults
% using explicit options in \includegraphics[width, height, ...]{}
\setkeys{Gin}{width=\maxwidth,height=\maxheight,keepaspectratio}
% Set default figure placement to htbp
\makeatletter
\def\fps@figure{htbp}
\makeatother
\setlength{\emergencystretch}{3em} % prevent overfull lines
\providecommand{\tightlist}{%
  \setlength{\itemsep}{0pt}\setlength{\parskip}{0pt}}
\setcounter{secnumdepth}{-\maxdimen} % remove section numbering
\ifLuaTeX
  \usepackage{selnolig}  % disable illegal ligatures
\fi
\IfFileExists{bookmark.sty}{\usepackage{bookmark}}{\usepackage{hyperref}}
\IfFileExists{xurl.sty}{\usepackage{xurl}}{} % add URL line breaks if available
\urlstyle{same}
\hypersetup{
  pdftitle={Joshua M. Rosenberg},
  hidelinks,
  pdfcreator={LaTeX via pandoc}}

\title{Joshua M. Rosenberg}
\author{}
\date{\vspace{-2.5em}}

\begin{document}
\maketitle

\hypertarget{curriculum-vitae}{%
\section{Curriculum Vitae}\label{curriculum-vitae}}

\hypertarget{contact-information}{%
\section{Contact Information}\label{contact-information}}

Associate Professor, STEM Education\\
Department of Theory and Practice in Teacher Education\\
Faculty Fellow, Center for Enhancing Education in Mathematics and
Sciences\\
The University of Tennessee, Knoxville\\
420 Claxton, 1122 Volunteer Blvd., Knoxville, TN 37996\\
865-974-5973 \textbar{} \url{jmrosenberg@utk.edu}\\
Homepage: \url{https://joshuamrosenberg.com}\\
Google Scholar:
\url{https://scholar.google.com/citations?hl=en\&user=nxVowRQAAAAJ}

\hypertarget{research-interests}{%
\section{Research Interests}\label{research-interests}}

Educational data science, science education, educational technology

\hypertarget{education}{%
\section{Education}\label{education}}

\hypertarget{degrees}{%
\subsubsection{Degrees}\label{degrees}}

2018, PhD, Educational Psychology \& Educational Technology\\
Michigan State University

2012, MA, Education\\
Michigan State University

2010, BS, Biology\\
University of North Carolina, Asheville

\hypertarget{additional-qualifications}{%
\subsubsection{Additional
Qualifications}\label{additional-qualifications}}

2016, Graduate Certificate, Science Education\\
Michigan State University

2010, Educator's License - Science and Biology, Teacher Licensure
Program\\
University of North Carolina, Asheville

\hypertarget{professional-experience}{%
\section{Professional Experience}\label{professional-experience}}

2023-present, Associate Professor, STEM Education\\
University of Tennessee, Knoxville

2018-2023, Assistant Professor, STEM Education\\
University of Tennessee, Knoxville

2012-2018, Graduate Research and Teaching Assistant\\
Michigan State University

2010-2012, Science Teacher (Biology and Earth Science)\\
Shelby High School, Shelby, NC

2009-2010, Science Teacher Intern (Biology and Chemistry)\\
C.D. Owen High School, Swannanoa, NC

\hypertarget{selected-grants}{%
\section{Selected Grants}\label{selected-grants}}

\hypertarget{pi-co-pi-and-co-i}{%
\subsection{PI, Co-PI, and Co-I}\label{pi-co-pi-and-co-i}}

2023-2028, PI, \emph{CAREER: Creatively Reimagining Engagements with
Data in Biology Learning Environments}, (\$846,612), NSF.
\href{https://www.nsf.gov/awardsearch/showAward?AWD_ID=2239152\&HistoricalAwards=false}{NSF
Grant No.~2239152}.

2022-2025, Co-PI, \emph{Computer Science for Appalachia: Expanding a
research-practice partnership to integrate computer science and literacy
in rural East Tennessee schools}, (\$999,980; with \emph{PI} Lynn
Hodge). NSF. NSF Grant No.~2219418.

2022-2024, Co-PI, \emph{Broadening participation in introductory
computer science: investigating self-assessment practices for increasing
student learning and self-efficacy in two institutional contexts}
(\$299,836; with \emph{PI} Alex Lishinski). NSF.
\href{https://www.nsf.gov/awardsearch/showAward?AWD_ID=2215245\&HistoricalAwards=false}{NSF
Grant No.~2215245}

2022-2023, Co-PI, \emph{Launching a Micro-credential in Educational Data
Analytics} (\$10,000; with \emph{Co-PI} Louis Rocconi). University of
Tennessee, Knoxville's College of Education, Health, and Human Sciences
Strategic Investment Program.

2018-2020, Co-PI, \emph{Exploring how beginning elementary mathematics
teachers seek out resources through social media} (\$8,820; \emph{PI}:
Stephen Aguilar). Herman \& Rasiej K-5 Mathematics Initiative,
University of Southern California.

\hypertarget{senior-personnel}{%
\subsection{Senior Personnel}\label{senior-personnel}}

2020-2023, Senior Personnel, \emph{Learning analytics in STEM education
research institute} (\$933,150; \emph{PI}, Shaun Kellogg, North Carolina
State University; UTK subcontract: \$62,870. National Science Foundation
(NSF),
\href{https://www.nsf.gov/awardsearch/showAward?AWD_ID=2025090\&HistoricalAwards=false}{NSF
Grant No.~2025090}

2019-2022, Senior Personnel, \emph{Medical entomology and geospatial
analyses: Bringing innovation to teacher education and surveillance
studies} (\$149,611; \emph{PI}: Rebecca Trout Fryxell). United States
Department of Agriculture - Agriculture and Food Research Initiative.
(USDA Grant No.~2019-68010-29119) \url{https://www.megabitess.org/}

\hypertarget{fellowships-and-awards}{%
\section{Fellowships and Awards}\label{fellowships-and-awards}}

2023, Research Worth Reading Award, Research in Artificial Intelligence
in Science Eduation Research Interest Group, National Association for
Research in Science Teaching

2023, Outstanding Graduate Research Mentor Award, Graduate Student
Senate, University of Tennessee, Knoxville

2022, Early Career Award, Technology as an Agent of Change in Teaching
and Learning (TACTL) Special Interest Group (SIG), American Educational
Research Association (AERA)

2021, Best Poster Award, Fourteenth International Conference on
Educational Data Mining

2021-2022, Open Educational Resources (OER) Research Fellow, William and
Flora Hewlett Foundation

2021, Louie M. \& Betty M. Phillips Faculty Support in Education Award,
University of Tennessee, Knoxville (UTK)

2021, Mentor, Summer Undergraduate Research Internship Program, Office
of Undergraduate Research, UTK

2020, Research Assistant Award, Office of Undergraduate Research, UTK

2020, Southeastern Conference (SEC) Visiting Faculty Travel Grant
Program (Host: Annelise Russell, Martin School of Public Policy,
University of Kentucky)

2019-2020, Initiative for the Future Faculty Development Program, UTK

2019, Finalist, Association for Science Teacher Education John C. Park
National Technology Leadership Institute Fellowship

2017, Delia Koo Global Travel Fellowship, Michigan State University
(MSU)

2017, Michigan Virtual Learning Research Institute Dissertation
Fellowship

2017, Concord Consortium Data Science Educational Technology Fellowship

2017, Council of Graduate Students Disciplinary Leadership Award, MSU

2016, College of Education Alumni Fellowship, MSU

2016, Best Paper Award, Technological Pedagogical Content Knowledge SIG,
Society for Information Technology and Teacher Education International
Conference

2015, Cotterman Family Endowment for Education Summer Research
Fellowship, MSU

2014, Outstanding Paper Award, Society for Information Technology and
Teacher Education International Conference

2013, Massive Open Online Course Research and Development Fellowship,
MSU

2009-2012, Burroughs Wellcome Fund Scholar, University of North
Carolina, Asheville

\hypertarget{publications}{%
\section{Publications}\label{publications}}

+ Denotes a collaboration with a mentee who is a graduate student\\
\^{} Denotes a collaboration with a mentee who is an undergraduate
student

\hypertarget{books}{%
\subsection{Books}\label{books}}

\hangindent=2em Rosenberg, K., \& Rosenberg, J. M. (under contract).
\emph{Little kids, big adventures: Exploring the Cumberland Plateau,
Tennessee Valley, and Great Smoky Mountains with kids}. University of
Tennessee Press. \url{https://littlekidsbigadventures.com/}

\hangindent=2em Estrellado, R. A., Freer, E. A., Mostipak, J.,
Rosenberg, J. M., \& Velásquez, I. C. (2020). \emph{Data science in
education using R}. Routledge. \emph{Note.} All authors contributed
equally. \url{http://www.datascienceineducation.com/}

\hypertarget{selected-articles-published-in-refereed-journals}{%
\subsection{Selected Articles Published in Refereed
Journals}\label{selected-articles-published-in-refereed-journals}}

\hangindent=2em Rosenberg, J.M. (in press). Open and Useful? Exploring
the Science Education Resources on OER Commons. \emph{Contemporary
Issues in Technology and Teacher Education}.
\url{https://osf.io/preprints/9adhp/}

\hangindent=2em Carpenter, J. P., Morrison, S. A., Rosenberg, J. M., \&
Hawthorne, K. A. (advance online publication). Using social media in
pre-service teacher education: The case of a program-wide twitter
hashtag. \emph{Teaching and Teacher Education, 124}, 1-17.

\hangindent=2em Rosenberg, J., \^{}Borchers, C., Stegenga, S. M.,
\^{}Burchfield, M. A., Anderson, D., \& Fischer, C. (2022). How
educational institutions reveal students' personally identifiable
information on Facebook. \emph{Learning, Media, \& Technology.}
\url{https://www.tandfonline.com/doi/full/10.1080/17439884.2022.2140672}

\hangindent=2em Rosenberg, J. M., \^{}Borchers, C., \^{}Burchfield, M.
A., Anderson, D., Stegenga, S. M., \& Fischer, C. (2022). Posts About
Students on Facebook: A Data Ethics Perspective. \emph{Educational
Researcher, 51}(8), 547-550.

\hangindent=2em Kubsch, M., Krist, C., \& Rosenberg, J. M. (2022).
Distributing Epistemic Functions and Tasks - A framework for augmenting
human analytic power with machine learning in science education
research. \emph{Journal of Research in Science Teaching}.
\url{https://onlinelibrary.wiley.com/doi/full/10.1002/tea.21803}.
\emph{Note.} All authors contributed equally. This paper received the
2023 Research Worth Reading Award for the Research in Artificial
Intelligence in Science Eduation Research Interest Group, National
Association for Research in Science Teaching.

\hangindent=2em Rosenberg, J. M., Kubsch, M., Wagenmakers, E.-J., \&
Dogucu, M. (2022). Making sense of uncertainty in the science classroom:
A Bayesian approach. \emph{Science \& Education}, 31, 1239--1262.
\url{https://link.springer.com/article/10.1007/s11191-022-00341-3}

\hangindent=2em Jones. R. S., \& Rosenberg, J. M. (2022). Characterizing
whole class discussions about data and statistics with conversation
profile analysis. \emph{Journal of Mathematical Behavior, 67}, 1-16.
\url{https://www.sciencedirect.com/science/article/abs/pii/S0732312322000645}

\hangindent=2em Rosenberg, J. M., Schultheis, E., Kjelvik, M., Reedy,
A., \& +Sultana, O. (2022). Big data, big changes? A survey of K-12
science teachers in the United States on which data sources and tools
they use in the classroom. \emph{British Journal of Educational
Technology, 53}(5), 1179-1201.
\url{https://bera-journals.onlinelibrary.wiley.com/doi/10.1111/bjet.13245}

\hangindent=2em +Michela, E., Rosenberg, J., Kimmons, R., +Sultana, O.,
\^{}Burchfield, M. A., \& \^{}Thomas, T. (2022). ``We are trying to
communicate the best we can'': Understanding districts' communication on
Twitter during the COVID-19 pandemic. \emph{AERA Open, 8}, 1-18.
\url{https://doi.org/10.1177/23328584221078542}

\hangindent=2em Trout Fryxell, R. T., Camponovo, M., Smith, B.,
Butefish, K., Rosenberg, J. M., Andsager, J. L., \ldots{} \& Willis, M.
P. (2022). Development of a community-driven mosquito surveillance
program for vectors of La Crosse virus to educate, inform, and empower a
community. \emph{Insects, 13}(2), 164.
\url{https://www.mdpi.com/2075-4450/13/2/164}

\hangindent=2em Rutherford, T., Duck, K., Rosenberg, J. M., \& Patt, R.
(2022). Leveraging mathematics software data to understand student
learning and motivation during the COVID-19 pandemic. \emph{Journal of
Research on Technology in Education, 54}(1), 94-131.
\url{https://www.tandfonline.com/doi/full/10.1080/15391523.2021.1920520}

\hangindent=2em Aguilar, S. J., Rosenberg, J., Greenhalgh, S., Fütterer,
T., Lishinski, A., \& Fischer, C. (2021). A different experience in a
different moment? Teachers' social media use before and during the
COVID-19 pandemic. \emph{AERA Open, 7}, 1-17.
\url{https://journals.sagepub.com/doi/full/10.1177/23328584211063898}

\hangindent=2em +Lawson, M. A., Herrick, I., R., \& Rosenberg, J. M.
(2021). Better together: Mathematics and science pre-service teachers'
sensemaking about STEM. \emph{Educational Technology \& Society, 24}(4),
180--192.

\hangindent=2em Schweinsburg, M., \ldots{} Luis, S. (2021). Same data,
different conclusions: Radical dispersion in empirical results.
\emph{Organizational Behavior and Human Decision Processes, 165}(7),
228-249.
\url{https://www.sciencedirect.com/science/article/pii/S0749597821000200}
(\emph{Note.} I was an author and contributor to this large-scale,
collaborative project.)

\hangindent=2em Greenhalgh, S. P., Rosenberg, J. M., \& Russell, A.
(2021). The influence of policy and context on teachers' social media
use. \emph{The British Journal of Educational Technology, 52}(5),
2020-2037.
\url{https://bera-journals.onlinelibrary.wiley.com/doi/10.1111/bjet.13096?af=R}

\hangindent=2em Frank, K. A., Lin, Q., Maroulis, S., Strassman, A., Xu,
R., Rosenberg, J. M., Hayter, C., Mahmoud, R., Kolak, M., Dietz, T., \&
Zhang, L. (2021). Hypothetical case replacement can be used to quantify
the robustness of trial results. \emph{Journal of Clinical Epidemiology,
134}(6), 150-159.
\url{https://www.sciencedirect.com/science/article/pii/S0895435621000366}
\emph{Note.} I was a scientific programmer for this project.

\hangindent=2em Rosenberg, J. M., \^{}Borchers, C., Dyer, E., Anderson,
D. J., \& Fischer, C. (2021). Advancing new methods for understanding
public sentiment about educational reforms: The case of Twitter and the
Next Generation Science Standards. \emph{AERA Open, 7}, 1-17.
\url{https://journals.sagepub.com/doi/10.1177/23328584211024261}

\hangindent=2em Ranellucci, J., Robinson, K., Rosenberg, J. M., Lee,
Y.-K., Roseth, C., \& Linnenbrink-Garcia. (2021). Comparing the roles
and correlates of emotions in class and during online video lectures in
a flipped anatomy classroom. \emph{Contemporary Educational Psychology,
64}(4), 1-15. \url{https://doi.org/10.1016/j.cedpsych.2021.101966}

\hangindent=2em Akcaoglu, M., Rosenberg, J. M., Hodges, C. B., Hilpert,
J. (2021). An exploration of factors impacting middle school students'
attitudes toward computer programming. \emph{Computers in the Schools,
38}(1), 19-35. \url{https://doi.org/10.1080/07380569.2021.1882209}

\hangindent=2em Rosenberg, J. M., \& Krist, C. (2021). Combining machine
learning and qualitative methods to elaborate students' ideas about the
generality of their model-based explanations. \emph{Journal of Science
Education and Technology, 30}(2), 255-267.
\url{https://link.springer.com/article/10.1007\%2Fs10956-020-09862-4}.
\emph{Note.} Both authors contributed equally.

\hangindent=2em Rosenberg, J. M., \& Staudt Willet, K. B. (2021).
Balancing participants' privacy and open science in the context of
COVID-19: A response to Ifenthaler \& Schumacher (2016).
\emph{Educational Technology Research \& Development, 69}(1), 347-351.

\hangindent=2em Harper, F. K., Rosenberg, J. M., \^{}Comperry, S.,
\^{}Howell, K., \& \^{}Womble, S. (2021). \#Mathathome during the
COVID-19 Pandemic: Exploring and reimagining resources and social
supports for parents. \emph{Education Sciences, 11}(2), 60, 1-24.
\url{https://www.mdpi.com/2227-7102/11/2/60}

\hangindent=2em Anderson, D. J., Rowley, B., Stegenga, S., Irvin, P. S.,
\& Rosenberg, J. M. (2020). Evaluating content-related validity evidence
using a text-based, machine learning procedure. \emph{Educational
Measurement: Issues and Practice, 39}(4), 53-64.
\url{https://onlinelibrary.wiley.com/doi/abs/10.1111/emip.12314}

\hangindent=2em Rosenberg, J. M., Reid, J., Dyer, E., Koehler, M. J.,
Fischer, C., \& McKenna, T. J. (2020). Idle chatter or compelling
conversation? The potential of the social media-based \#NGSSchat network
as a support for science education reform efforts. \emph{Journal of
Research in Science Teaching, 57}(9), 1322-1355.
\url{https://onlinelibrary.wiley.com/doi/10.1002/tea.21660}

\hangindent=2em Carpenter, J., Rosenberg, J. M., Dousay, T.,
Romero-Hall, E., Trust, T., Kessler, A., Phillips, M., Morrison, S.,
Fischer, C. \& Krutka, D. (2020). What should teacher educators know
about technology? Perspectives and self-assessments. \emph{Teaching and
Teacher Education, 95}(10), 103-124.
\url{https://doi.org/10.1016/j.tate.2020.103124}

\hangindent=2em Ranellucci, J., Rosenberg, J. M., \& Poitras, E. (2020).
Exploring pre-service teachers' use of technology: The technology
acceptance model and expectancy-value theory. \emph{Journal of Computer
Assisted Learning, 36}(6), 810-824.
\url{http://dx.doi.org/10.1111/jcal.12459}

\hangindent=2em Schmidt, J. A., Beymer, P. N., Rosenberg, J. M.,
Naftzger, N. J., \& Shumow, L. (2020). Experiences, activities, and
personal characteristics as predictors of engagement in STEM-focused
summer programs. \emph{Journal of Research in Science Teaching, 57}(8),
1281-1309.
\url{https://onlinelibrary.wiley.com/doi/full/10.1002/tea.21630}

\hangindent=2em Greenhalgh, S. P., Rosenberg, J. M., Koehler, M. J.,
Akcaoglu, M., \& Staudt Willet, K. B. (2020). Identifying multiple
learning spaces within a single teacher-focused Twitter hashtag.
\emph{Computers \& Education, 148}(4).
\url{https://doi.org/10.1016/j.compedu.2020.103809}

\hangindent=2em Beymer, P. N., Rosenberg, J. M., \& Schmidt, J. A.
(2020). Does choice matter or is it all about interest? An investigation
using an experience sampling approach in high school science classrooms.
\emph{Learning and Individual Differences, 78}(2), 1-15.
\url{https://doi.org/10.1016/j.lindif.2019.101812}

\hangindent=2em Rosenberg, J. M., +Edwards, A., \& Chen, B. (2020).
Getting messy with data: Tools and strategies to help students analyze
and interpret complex data sources. \emph{The Science Teacher, 87}(5).
\url{https://learningcenter.nsta.org/resource/?id=10.2505/4/tst20_087_05_30}

\hangindent=2em Xu, R., Frank, K. A., Maroulis, S., \& Rosenberg, J. M.
(2019). konfound: Command to quantify robustness of causal inferences.
\emph{The Stata Journal, 19}(3), 523-550.
\url{https://journals.sagepub.com/doi/full/10.1177/1536867X19874223}

\hangindent=2em Blondel, D. V., Sansone, A., Rosenberg, J. M., Yang, B.
W., Linennbrink-Garcia, L., \& Schwarz-Bloom, R. D. (2019). Development
of an online experiment platform (Rex) for high school biology.
\emph{Journal of Formative Design for Learning, 3}(1) 62-81.
\url{https://www.ncbi.nlm.nih.gov/pmc/articles/PMC6716597/}

\hangindent=2em Henriksen, D., Mehta, R. \& Rosenberg, J. (2019).
Supporting a creatively focused technology fluent mindset among
educators: survey results from a five-year inquiry into teachers'
confidence in using technology. \emph{Journal of Technology and Teacher
Education, 27}(1), 63-95.
\url{https://www.learntechlib.org/primary/p/184724/}

\hangindent=2em Rosenberg, J. M., \& +Lawson, M. J. (2019). An
investigation of students' use of a computational science simulation in
an online high school physics class. \emph{Education Sciences, 9}(49),
1-19. \url{https://www.mdpi.com/2227-7102/9/1/49}

\hangindent=2em Rosenberg, J. M., Beymer, P. N., Anderson, D. J., \&
Schmidt, J. A. (2018). tidyLPA: An R package to easily carry out Latent
Profile Analysis (LPA) using open-source or commercial software.
\emph{Journal of Open Source Software, 3}(30), 978.
\url{https://doi.org/10.21105/joss.00978}

\hangindent=2em Greenhalgh, S. P., Staudt Willet, K. B., Rosenberg, J.
M., \& Koehler, M. J. (2018). Tweet, and we shall find: Using digital
methods to locate participants in educational hashtags.
\emph{TechTrends, 62}(5), 501-508.
\url{https://doi.org/10.1007/s11528-018-0313-6}

\hangindent=2em Beymer, P. N., Rosenberg, J. M., Schmidt, J. A., \&
Naftzger, N. (2018). Examining relationships between choice, affect, and
engagement in out-of-school time STEM programs. \emph{Journal of Youth
and Adolescence, 47}(6), 1178-1191.
\url{https://doi.org/10.1007/s10964-018-0814-9}

\hypertarget{presentations}{%
\section{Presentations}\label{presentations}}

\hypertarget{invited-talks-5}{%
\subsection{Invited Talks (5)}\label{invited-talks-5}}

\hangindent=2em Hu, A. D., Greenhalgh, S. P. Rosenberg, J. M., \&
Staudt-Willet, B. (March, 2023). \emph{What ChatGPT is, how it is
impacting universities, and how might we make ``good'' use of it.}.
Panel presentation at the Michigan State University College of
Education. Michigan State University, East Lansing, MI.

\hangindent=2em Rosenberg, J. M. (April, 2021). \emph{AI and ML and
data! Oh my! Supporting teachers' and learners' work by considering the
human sides of data science}. Keynote presentation at the LEAD Graduate
School and Research Network retreat. The University of Tübingen,
Baden-Württemberg, Germany.

Rosenberg, J. M. (October, 2021). \emph{All together now: Leveraging
data science techniques alongside traditional approaches to understand
learning}. Invited presentation at the International Conference on
Education Research. Seoul National University, Seoul, South Korea.

\hangindent=2em Rosenberg, J. M. (February, 2020). \emph{Studying
education-focused Twitter hashtags in light of state-based and national
policies and practices}. Presentation at the 2020 Spring Seminar Series
at the Martin School of Public Policy at the University of Kentucky,
Lexington, KY.

\hangindent=2em Rosenberg, J.M. (September, 2019). \emph{Making data
science education count: Exploring the integration of data science into
education}. Presentation at the Middle Tennessee State University
Mathematics and Science Education Doctoral Seminar series. Middle
Tennessee State University, Murfreesboro, TN.

\hangindent=2em Rosenberg, J. M. (February, 2019). \emph{Making sense of
recent advances in the Technological Pedagogical Content Knowledge
framework}. English International Congress at the Universidad Técnica
del Norte, Ibarra, Ecuador.

\hypertarget{podcast}{%
\subsubsection{Podcast}\label{podcast}}

2021-, Co-host, \emph{About Practice} podcast,
\url{https://anchor.fm/about-practice}

2018-2019, Co-host, \emph{Impodster Syndrome} podcast,
\url{https://drive.google.com/drive/folders/1fwSaEKt9QzJPUlf-CYVVwPgN-pKBaAkW?usp=sharing}

\hypertarget{public-engagement}{%
\subsubsection{Public Engagement}\label{public-engagement}}

Dyer, E. B., Reid, J., \& Rosenberg, J. M. (2021, January 7). Science
Education \& Democracy {[}Synchronous Twitter Chat{]}. \#NGSSchat.

Dyer, E. B., McKenna, T. J., \& Rosenberg, J. M. (2020, December 3). Who
are the Experts Here? {[}Synchronous Twitter Chat{]}. \#NGSSchat.

Rosenberg, J. M., McKenna, T. J., \& Dyer, E. B. (2020, June 18).
Exploring the Impact of Our Community {[}Synchronous Twitter Chat{]}.
\#NGSSchat. \url{https://wke.lt/w/s/rqTNhr}

\hypertarget{professional-affiliations}{%
\subsubsection{Professional
Affiliations}\label{professional-affiliations}}

American Educational Research Association, 2012 - Present\\
International Society of the Learning Sciences, 2014 - Present\\
Learning Analytics \& Knowledge, 2020 - Present\\
National Association for Research in Science Teaching, 2015 - Present

\end{document}

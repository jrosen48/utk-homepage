%!TEX TS-program = xelatex
%!TEX encoding = UTF-8 Unicode
% Awesome CV LaTeX Template for CV/Resume
%
% This template has been downloaded from:
% https://github.com/posquit0/Awesome-CV
%
% Author:
% Claud D. Park <posquit0.bj@gmail.com>
% http://www.posquit0.com
%
%
% Adapted to be an Rmarkdown template by Mitchell O'Hara-Wild
% 23 November 2018
%
% Template license:
% CC BY-SA 4.0 (https://creativecommons.org/licenses/by-sa/4.0/)
%
%-------------------------------------------------------------------------------
% CONFIGURATIONS
%-------------------------------------------------------------------------------
% A4 paper size by default, use 'letterpaper' for US letter
\documentclass[11pt, a4paper]{awesome-cv}

% Configure page margins with geometry
\geometry{left=1.4cm, top=.8cm, right=1.4cm, bottom=1.8cm, footskip=.5cm}

% Specify the location of the included fonts
\fontdir[fonts/]

% Color for highlights
% Awesome Colors: awesome-emerald, awesome-skyblue, awesome-red, awesome-pink, awesome-orange
%                 awesome-nephritis, awesome-concrete, awesome-darknight

\colorlet{awesome}{awesome-red}

% Colors for text
% Uncomment if you would like to specify your own color
% \definecolor{darktext}{HTML}{414141}
% \definecolor{text}{HTML}{333333}
% \definecolor{graytext}{HTML}{5D5D5D}
% \definecolor{lighttext}{HTML}{999999}

% Set false if you don't want to highlight section with awesome color
\setbool{acvSectionColorHighlight}{true}

% If you would like to change the social information separator from a pipe (|) to something else
\renewcommand{\acvHeaderSocialSep}{\quad\textbar\quad}

\def\endfirstpage{\newpage}

%-------------------------------------------------------------------------------
%	PERSONAL INFORMATION
%	Comment any of the lines below if they are not required
%-------------------------------------------------------------------------------
% Available options: circle|rectangle,edge/noedge,left/right

\name{Joshua}{Rosenberg}

\position{Assistant Professor, STEM Education}

\email{\href{mailto:jmrosen48@gmail.com}{\nolinkurl{jmrosen48@gmail.com}}}
\homepage{joshuamrosenberg.com}
\github{jrosen48}
\twitter{jrosenberg6432}

% \gitlab{gitlab-id}
% \stackoverflow{SO-id}{SO-name}
% \skype{skype-id}
% \reddit{reddit-id}


\usepackage{booktabs}

\providecommand{\tightlist}{%
	\setlength{\itemsep}{0pt}\setlength{\parskip}{0pt}}

%------------------------------------------------------------------------------



% Pandoc CSL macros
\newlength{\cslhangindent}
\setlength{\cslhangindent}{1.5em}
\newlength{\csllabelwidth}
\setlength{\csllabelwidth}{3em}
\newenvironment{CSLReferences}[3] % #1 hanging-ident, #2 entry spacing
 {% don't indent paragraphs
  \setlength{\parindent}{0pt}
  % turn on hanging indent if param 1 is 1
  \ifodd #1 \everypar{\setlength{\hangindent}{\cslhangindent}}\ignorespaces\fi
  % set entry spacing
  \ifnum #2 > 0
  \setlength{\parskip}{#2\baselineskip}
  \fi
 }%
 {}
\usepackage{calc}
\newcommand{\CSLBlock}[1]{#1\hfill\break}
\newcommand{\CSLLeftMargin}[1]{\parbox[t]{\csllabelwidth}{#1}}
\newcommand{\CSLRightInline}[1]{\parbox[t]{\linewidth - \csllabelwidth}{#1}}
\newcommand{\CSLIndent}[1]{\hspace{\cslhangindent}#1}

\begin{document}

% Print the header with above personal informations
% Give optional argument to change alignment(C: center, L: left, R: right)
\makecvheader

% Print the footer with 3 arguments(<left>, <center>, <right>)
% Leave any of these blank if they are not needed
% 2019-02-14 Chris Umphlett - add flexibility to the document name in footer, rather than have it be static Curriculum Vitae
\makecvfooter
  {December 2020}
    {Joshua Rosenberg~~~·~~~Curriculum Vitae}
  {\thepage}


%-------------------------------------------------------------------------------
%	CV/RESUME CONTENT
%	Each section is imported separately, open each file in turn to modify content
%------------------------------------------------------------------------------



\hypertarget{highlights}{%
\section{Highlights}\label{highlights}}

\begin{itemize}
\tightlist
\item
  Experienced computational social scientist and data scientist
\item
  Develop of open-source software, including three R packages on CRAN
\item
  Manager of an educational data science research group with three NSF
  grants
\end{itemize}

\hypertarget{experience}{%
\section{Experience}\label{experience}}

\begin{cventries}
    \cventry{Assistant Professor, STEM Education}{University of Tennessee}{Knoxville, TN}{2018-Present}{\vspace{-4mm}}
\end{cventries}

\hypertarget{education}{%
\section{Education}\label{education}}

\begin{cventries}
    \cventry{PhD, Educational Psychology and Educational Technology}{Michigan State University}{East Lansing, MI}{2018}{\vspace{-4mm}}
    \cventry{MA, Education}{Michigan State University}{East Lansing, MI}{2014}{\vspace{-4mm}}
    \cventry{BS, Biology}{University of North Carolina, Asheville}{Asheville, NC}{2012}{\vspace{-4mm}}
\end{cventries}

\hypertarget{external-funding}{%
\section{External Funding}\label{external-funding}}

\hypertarget{grants}{%
\section{Grants}\label{grants}}

2019-2021, Principle Investigator (PI), \emph{Understanding the
development of interest in computer science: An experience sampling
approach} (\$346,688). National Science Foundation {[}NSF{]}.
\url{http://picsul.utk.edu/} (NSF Grant
No.~\href{https://www.nsf.gov/awardsearch/showAward?AWD_ID=1937700\&HistoricalAwards=false}{1937700})

2019-2021, Co-PI, \emph{CS for Appalachia: A research-practice
partnership for integrating computer science into East Tennessee
schools} (\$252,453; \emph{PI}: Lynn Hodge, University of Tennessee,
Knoxville). NSF. (NSF Grant
No.~\href{https://www.nsf.gov/awardsearch/showAward?AWD_ID=1923509\&HistoricalAwards=false}{1923509})

2019-2022, Co-PI, \emph{Advancing computational grounded theory for
audiovisual data from STEM classrooms} (\$1,313,855; \emph{PI}:
Christina Krist, University of Illinois Urbana-Champaign; University of
Tennessee, Knoxville subcontract: \$101,469). NSF.
\url{https://tca2.education.illinois.edu/} (NSF Grant
No.~\href{https://www.nsf.gov/awardsearch/showAward?AWD_ID=1920796\&HistoricalAwards=false}{1920796})

2019-2020, PI, \emph{Planting the seeds for computer science education
in East Tennessee through a research-practice partnership} (\$13,200).
Community Engaged Research Seed Program, University of Tennessee,
Knoxville.

\hypertarget{select-publications}{%
\section{Select Publications}\label{select-publications}}

\hangindent=2em Estrellado, R. A., Freer, E. A., Mostipak, J.,
Rosenberg, J. M., \& Velásquez, I. C. (in press). Data science in
education using R. London, England: Routledge. Nb. All authors
contributed equally. \url{http://www.datascienceineducation.com/}

\hangindent=2em Rosenberg, J. M., Reid, J., Dyer, E., Koehler, M. J.,
Fischer, C., \& McKenna, T. J. (in press). Idle chatter or compelling
conversation? The potential of the social media-based \#NGSSchat network
as a support for science education reform efforts. \emph{Journal of
Research in Science Teaching}.

\hangindent=2em Anderson, D. J., Rowley, B., Stegenga, S., Irvin, P. S.,
\& Rosenberg, J. M. (advance online publication). Evaluating
content-related validity evidence using a text-based, machine learning
procedure. Educational Measurement: Issues and Practice.
\url{https://onlinelibrary.wiley.com/doi/abs/10.1111/emip.12314}

\hangindent=2em Greenhalgh, S. P., Rosenberg, J. M., Koehler, M. J.,
Akcaoglu, M., \& Staudt Willet, K. B. (2020). Identifying multiple
learning spaces within a single teacher-focused Twitter hashtag.
Computers \& Education, 148(4).
\url{https://doi.org/10.1016/j.compedu.2020.103809}

\hangindent=2em Xu, R., Frank, K. A., Maroulis, S., \& Rosenberg, J. M.
(2019). konfound: Command to quantify robustness of causal inferences.
The Stata Journal, 19(3), 523-550.

\hangindent=2em Rosenberg, J. M., van Lissa, C. J., Beymer, P. N.,
Anderson, D. J., Schell, M. J. \& Schmidt, J. A. (2019). tidyLPA: Easily
carry out Latent Profile Analysis (LPA) using open-source or commercial
software {[}R package{]}. \url{https://data-edu.github.io/tidyLPA/}

\emph{See all publications here}:
\url{https://joshuamrosenberg.com/about/\#publications}

\hypertarget{select-presentations}{%
\section{Select Presentations}\label{select-presentations}}

\hangindent=2em Rosenberg, J. M., Qinyun, L., Xu, R., Maroulis, S., \&
Frank, K. A. (July, 2020). \emph{The konfound R package and Shiny app
for robustness analysis}. Presentation at the useR conference,
St.~Louis, MO.

\hangindent=2em Rosenberg, J. M., Beymer, P. N., Phun, V., Schmidt, J.
A. (2020, April). Sources of variability for students' engagement in
science: Findings from a cross-classified, multivariate modeling
approach. In P. N. Beymer, D. K. Benden, \& M. L. Bernacki (Chairs),
\emph{Affordances and modeling of intensive data}. Symposium conducted
at the American Educational Research Association Annual Meeting, San
Francisco, CA.

\hangindent=2em Rosenberg, J. M, Beymer, P. N., Houslay, T. M., \&
Schmidt, J. A. (2019, April). Using a multivariate, multi-level model to
understand how youths' in-the-moment engagement predicts changes in
youths' interest. In M. Bernacki, A. Kaplan, and L. Linnenbrink-Garcia
(Chairs), \emph{Embracing and modeling the complex dynamics of
motivation and engagement: Contextual, temporal, dynamic, and
systematic}. Symposium conducted at the Annual Meeting of the American
Educational Research Association, Toronto, CA.

\emph{See all presentations here}:
\url{https://joshuamrosenberg.com/about/\#presentations}

\hypertarget{software-developed}{%
\section{Software Developed}\label{software-developed}}

\hypertarget{r-packages-on-the-comprehensive-r-archive-network-cran}{%
\subsection{R packages on the Comprehensive R Archive Network
(CRAN)}\label{r-packages-on-the-comprehensive-r-archive-network-cran}}

Rosenberg, J. M., van Lissa, C. J., Beymer, P. N., Anderson, D. J.,
Schell, M. J. \& Schmidt, J. A. (2019). \emph{tidyLPA: Easily carry out
Latent Profile Analysis (LPA) using open-source or commercial software}
{[}R package{]}. \url{https://data-edu.github.io/tidyLPA/}

Rosenberg, J. M., Xu, R., \& Frank, K. A. (2019). \emph{konfound:
Quantify the robustness of causal inferences} {[}R package{]}.
\url{https://jrosen48.github.io/konfound/}

Rosenberg, J. M., Schmidt, J. A., Beymer, P. N., \& Steingut, R. (2018).
\emph{prcr: Person-Centered Analysis} {[}R package{]}.
\url{https://CRAN.R-project.org/package=prcr}

\hypertarget{r-packages-on-github}{%
\subsection{R packages on GitHub}\label{r-packages-on-github}}

Estrellado, R. A., Bovee, E. A., Mostipak, J., Rosenberg, J. M., \&
Velásquez, I. C. (2019). \emph{dataedu: Package for Data Science in
Education Using R}. \url{https://github.com/data-edu/dataedu}

Anderson, D. Heiss, A., and Rosenberg, J. M. (2019). \emph{equatiomatic:
Transform Models into LaTeX Equations.}
\url{https://github.com/datalorax/equatiomatic}

\hypertarget{interactive-web-application}{%
\subsection{Interactive Web
Application}\label{interactive-web-application}}

Rosenberg, J. M., Xu, R., \& Frank, K. A. (2019). \emph{Konfound-It!:
Quantify the robustness of causal inferences.}
\url{http://konfound-it.com}.

\hypertarget{web-application}{%
\subsection{Web Application}\label{web-application}}

Lishinski, A., \& Rosenberg, J. M. (2019). \emph{Short message survey:
An open-source, text-message based application for the experience
sampling method.} \url{https://github.com/picsul/short-message-survey}

\end{document}
